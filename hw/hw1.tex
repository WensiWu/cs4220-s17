\documentclass[12pt, leqno]{article}
\usepackage{amsfonts}
\usepackage{amsmath}
\usepackage{amssymb}
\usepackage{fancyhdr}
\usepackage{hyperref}
\usepackage{tikz}
\usepackage{pgfplots}
\usepackage{listings}

\newcommand{\bbR}{\mathbb{R}}
\newcommand{\bbC}{\mathbb{C}}
\newcommand{\calV}{\mathcal{V}}
\newcommand{\calW}{\mathcal{W}}
\newcommand{\ddiag}{\operatorname{diag}}
\newcommand{\fl}{\operatorname{fl}}
\newcommand{\macheps}{\epsilon_{\mathrm{mach}}}
\newcommand{\matlab}{\textsc{Matlab}}

\newcommand{\hdr}[2]{
  \pagestyle{fancy}
  \lhead{Bindel, Spring 2016}
  \rhead{Numerical Analysis (CS 4220)}
  \fancyfoot{}
  \begin{center}
    {\large{\bf Notes for #1}}
  \end{center}
  \lstset{language=matlab,columns=flexible}  
}


\begin{document} \hdr{HW 1}{Fri, Feb 5}

The first two problems should be plausible given what you know as of
Jan 29.  Ideally, the material presented on Monday, Feb 1 will allow
you to do the other two problems.  Don't be shy about asking for help
in office hours or on Piazza!

\paragraph*{1: Placing parens}
Suppose $A, B \in \bbR^{n \times n}$ are square matrices,
$D = \operatorname{diag}(d) \in \bbR^{n \times n}$ is a diagonal matrix,
and $u, v \in \bbR^n$ are vectors.  Write short
fragments of MATLAB to evaluate them as efficiently
as possible, and give the complexity in terms of $n$:
\begin{enumerate}
\item $v^T (I + D A D) v$
\item $u^T A^2 v$
\item $\operatorname{tr}(uv^T A)$
\end{enumerate}

\paragraph*{2: Recognizing rank}
Consider the MATLAB fragment

\lstset{language=matlab,frame=lines,columns=flexible}
\begin{lstlisting}
function [y] = hw1mult(x)
  n = length(x);
  A = reshape(1:n^2, n, n);
  y = A*x;
\end{lstlisting}

\begin{enumerate}
\item What is $A$ for $n = 3$?
\item Show that $A$ has rank two (independent of $n$).
\item Rewrite {\tt hw1mult} so that it runs in $O(n)$ time.
\end{enumerate}

\paragraph*{3: Norms!}
\begin{enumerate}
\item
  Show that $x \mapsto \|x\|_1 + \|x\|_\infty$ is a norm.
\item
  The space $\mathcal{P}_3$ of polynomials with degree less than or
  equal to three has a norm $\|p\|$ given by
  \[
    \|p\|^2 = \int_{-1}^1 p(x)^2 \, dx
  \]
  For a general cubic $p(x) = ax^3 + bx^2 + cx + d$, write $\|p\|$
  in terms of $a, b, c, d$.
\end{enumerate}

\paragraph*{4: Pushing products}
Suppose $A \in \bbR^{n \times n}$ is symmetric and positive definite.  The $A$-norm of a
vector $v \in \bbR^{n}$ is $\|v\|_A = \sqrt{v^T A v}$.  Describe how to reconstruct
$A$ given a function that computes $\|v\|_A$ for any given vector.
Code it up in a function with the following interface:
\begin{lstlisting}
function [A] = hw1normA(normfun, n)
% Given a function to evaluate the Euclidean norm of a length n vector
% v with respect to the A inner product, reconstruct A.
\end{lstlisting}


\end{document}
