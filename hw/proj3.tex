\documentclass[12pt, leqno]{article} %% use to set typesize 
\usepackage{amsfonts}
\usepackage{amsmath}
\usepackage{amssymb}
\usepackage{fancyhdr}
\usepackage{hyperref}
\usepackage{tikz}
\usepackage{pgfplots}
\usepackage{listings}

\newcommand{\bbR}{\mathbb{R}}
\newcommand{\bbC}{\mathbb{C}}
\newcommand{\calV}{\mathcal{V}}
\newcommand{\calW}{\mathcal{W}}
\newcommand{\ddiag}{\operatorname{diag}}
\newcommand{\fl}{\operatorname{fl}}
\newcommand{\macheps}{\epsilon_{\mathrm{mach}}}
\newcommand{\matlab}{\textsc{Matlab}}

\newcommand{\hdr}[2]{
  \pagestyle{fancy}
  \lhead{Bindel, Spring 2016}
  \rhead{Numerical Analysis (CS 4220)}
  \fancyfoot{}
  \begin{center}
    {\large{\bf Notes for #1}}
  \end{center}
  \lstset{language=matlab,columns=flexible}  
}


\begin{document} \phdr{Proj 3: Evolutionary Dynamics}

\section{Introduction}

Nonlinear dynamics is a fascinating area for both theoretical
analysis and numerics.  Almost any corner of science or engineering
that considers how things evolve in time\footnote{%
That is, almost any corner of science or engineering.}
features some set of favorite nonlinear model problems
that illustrate interesting qualitative features.
In evolutionary game theory, an area connected to both
mathematical biology and economics, there are a wide variety
of such models.  These include the {\em replicator equations},
the {\em Lotka-Volterra equations}, and many others.  In this
project, we will consider a variant of the Lotka-Volterra
equations inspired by a recent paper by Toupo, Strogatz, Cohen,
and Rand\footnote{%
``Evolutionary game dynamics of controlled and automatic
  decision-making.''
  {\em Chaos}, vol 25.7 (2015), 071320; also available as
  \url{http://arxiv.org/abs/1507.01561}.
}.
I would be surprised if this model is totally new, but know
of no related work.  I would welcome any comments about related
models in the literature that you might come across as you
work on the project.

\subsection{Single strategy case}

We begin with a case that is simple enough to analyze by hand.  As a
motivating story, we consider a population of $x$ foragers searching
for food resources.  The quantity of food resources $r$ varies
depending on the action of the foragers.  We characterize foragers by
two closely related rates: a rate of reproduction $\alpha(r)$ and a
rate of consumption $\beta(r)$.  We assume $\beta(r) = 0$ for $r = 0$
and that $\beta$ is smooth and monotonically increasing; and for this
project, we will use $\alpha(r) = \beta(r)-\phi$ where $\phi > 0$ is
some basic level of resource consumption required for continued
comfortable existence; populations with less than $\phi$ resources per
individual decline, while populations with more grow.  The coupled
dynamics of the resource and the population are given by
\begin{align*}
  \dot{x} &= \alpha(r) x = (\beta(r)-\phi) x \\
  \dot{r} &= \rho - \beta(r) x.
\end{align*}

Since we have talked not at all about dynamics in this
class, we will focus on {\em equilibrium} solutions where
$\dot{x} = 0$ and $\dot{r} = 0$.  These correspond to solutions to
the nonlinear equations
\begin{align*}
  (\beta(r)-\phi) x & = 0 \\
  \rho - \beta(r) x & = 0.
\end{align*}
We assume $\rho > 0$ and $\beta(r) > 0$ for $r > 0$.  Under these
assumptions, an equilibrium (if one exists) must satisfy
\[
  x = \frac{\rho}{\phi} \quad \mbox{and} \quad
  \beta(r) = \phi.
\]

When an equilibrium exists, we analyze its {\em linear stability} by
looking at a Jacobian matrix at the equilibrium:
\[
\frac{\partial}{\partial (x,r)}
\begin{bmatrix} \dot{x} \\ \dot{r} \end{bmatrix} =
\begin{bmatrix}
  0 & \beta'(r) x \\
  -\beta(r) & -\beta'(r) x
\end{bmatrix}.
\]
Even knowing only that $\beta(r)$, $\beta'(r)$, and $x$ are all
positive numbers, we can tell from the structure of this matrix that
the eigenvalues have negative real part.  Hence, the equilibrium is
linearly stable.

\subsection{Competing strategies}

Now suppose our foragers follow varying strategies: some may be slower
but more systematic in seeking food, while others are quicker.  In
addition, we allow foragers to {\em mutate} or randomly change
strategies at some rate.  The system of differential equations
governing the system is
\begin{align*}
  \dot{x} &= \left(M + \operatorname{diag}(a(r))\right) x \\
  \dot{r} &= \rho - b(r)^T x
\end{align*}
where $r(t) \in \bbR$ and $\rho$ are as before and
\begin{itemize}
\item $r(t) \in \bbR$ is the amount of a shared resource at time $t$
\item $x(t) \in \bbR^n$ is the distribution of strategies in the system
\item $a(r) \in \bbR^n$ is a vector of growth rates for varying
  strategies
\item $b(r) \in \bbR^n$ is a vector of consumption rates for varying strategies
\item $M$ is a mutation matrix: $m_{ij}$ is a rate at which strategy
  $j$ mutates to strategy $i$.  Rows of $M$ must sum to zero.
\end{itemize}

What happens if there is no mutation?  In this case (usually), the
only equilibrium solutions will involve exactly one strategy (i.e.~one
$x_j$ is positive), and the solution will stable if $a_k(r) < 0$.
Hence, absent mutations or other terms that cause different strategies
to interact directly (instead of just through $r$), the competing
strategy case does not provide any interesting new equilibrium
behaviors beyond those that we see in the single-strategy case.

\subsection{Modeling resource acquisition}

So far, we have not discussed the specific form of $\beta(r)$.
We will use a model with one parameter, representing the carefulness
of the search.  Specifically, we assume that a search takes
some minimal time $\tau > 0$, and a search with time $\tau + \sigma$
succeeds\footnote{%
  To motivate this model, suppose we search a fraction $\sigma$
  of the space in time $\sigma$.  If there are $r$ uniformly
  distributed resources, the probability of encountering at least
  one is $1-(1-\sigma)^r \approx 1-\exp(-\sigma r)$, where the
  approximation holds for small $\sigma$.
}
with probability $1-\exp(-\sigma r)$.
With repeated searches, the expected rate of resource acquisition is
\[
  \beta(r; \sigma) = \frac{1-\exp(-\sigma r)}{\tau+\sigma},
\]
This has the desired behavior that $\beta(0) = 0$, and for any $r$
\[
  \beta'(r; \sigma) = \frac{\sigma}{\sigma + \tau} \exp(-\sigma r) > 0.
\]
As the number of resources grows large, we find that
$\beta$ approaches $1/(\sigma+\tau)$ from below.
We can also solve $\beta(r) = \phi$ in closed form:
\[
  r = -\frac{1}{\sigma} \log(1-\phi (\sigma + \tau)),
\]
assuming $(\sigma + \tau) \phi < 1$.  If $(\sigma + \tau) \phi \ll 1$,
a linearization of the log function gives the estimate
$r \approx \phi/(\sigma+\tau)$.
This makes sense physically, since if
$(\sigma + \tau) \phi > 1$ then the maximal rate of resource acquisition
possible $(1/(\sigma+\tau))$ is less than the minimal threshold to maintain
the population.

In the competing case, we will fix $\phi$ to be a constant for all
strategies, but choose varying search times $\sigma_j$ on
a strategy-by-strategy basis.

\section{Numerical treatment}

We now consider the case with $n$ possible strategies associated with
$b_j(r) = \beta(r; \sigma_j)$ and $a_j(r) = b_j(r)-\phi$ where
$\sigma_{j} = j (\phi^{-1}-\tau)/n$ for $j = 0, \ldots, n-1$.
We also allow a small uniform mutation
\[
  M = -\epsilon L, \quad \mbox{where} \quad L = I - \frac{1}{n} ee^T.
\]
We are interested in the equilibrium solutions, i.e.
\begin{align*}
  \left(M + \operatorname{diag}(a(r))\right)x &= 0\\
  \rho - b(r)^T x &= 0
\end{align*}
and their stability as a function of the mutation rate $\epsilon$, the
base resource consumption rate $\phi$, the resource production rate
$\rho$, and the basic search time $\tau$.  Our default values for
these parameters will be $\rho = 1$, $\phi = 2$, and $\tau = 0.1$;
unless otherwise stated, we will default to $n = 10^4$.

\subsection{No mutations}

We begin by considering the case where there is no mutation ($\epsilon
= 0$) and $n$ tends toward infinity.  In this case, we expect one
strategy associated with some specific value $\sigma \in \bbR$ to
dominate all others.  Hence, we want to find (as a function of $\phi$,
$\tau$, and $\rho$) values $\sigma_*$ and $r_*$ such that
$\beta(r_*; \sigma_*) = \phi$ and
$\beta(r_*; \sigma) > \beta(r_*; \sigma)$ for any $\sigma \neq \sigma_*$.

\paragraph{Task 1}
Write the following code to solve this problem numerically,
using any method you like:
\begin{lstlisting}[frame=single]
function [r, sigma] = opt_growth(phi, tau)
\end{lstlisting}
Check to see whether your solution is consistent with the behavior
of the numerical problem (you will generally only find one
stable equilibrium in the discrete problem).  Your solver should
be robust over the full range of values for the parameters;
if there is some corner where the problem becomes hard, you should
give a warning.

We can find the corresponding optimal strategy in the case of finite
$n$ by looking at the discrete points $\sigma_j$ that are closest to
$\sigma_*$  I recommend a function \verb|opt_growthn| to do this.

\subsection{Small mutations}

We expect the solutions to the equilibrium equations to depend
continuously (and usually differentiably) on $\epsilon$.
To understand the behavior near $\epsilon = 0$, we assume
$r = r(\epsilon)$ and $x = x(\epsilon)$ and differentiate
the equation
\[
  \left( -\epsilon L + \operatorname{diag}(a(r)) \right) x = 0
\]
with respect to $\epsilon$.

\paragraph{Task 2}
Write a routine to compute the derivative of the stable
equilibrium $(x,r)$ with respect to $\epsilon$ at $\epsilon = 0$.
\begin{lstlisting}[frame=single]
function [dr, dx] = small_mutation(phi, rho, tau, n)
\end{lstlisting}
Your solution algorithm should work even if $n$ is very large ---
don't use dense algorithms!

\subsection{Moderate mutations}

While the first-order perturbation theory is useful, it may not
be adequate to describe what happens as $\epsilon$ grows.  For this,
a Newton iteration with continuation may be useful; however, if you
have more clever solver ideas, you are welcome to use them.

\paragraph{Task 3}
Write a routine to continue the stable equilibrium solution
from $\epsilon = 0$ up to some specified $\epsilon_{\max}$.
Your routine should return the values of $\epsilon$ sampled
and the corresponding $x$ and $r$ values.
\begin{lstlisting}[frame=single]
function [epsilons, Xs, rs] = continuer_eps(eps_max, phi, rho, tau, n)
\end{lstlisting}
As before, you should not take $O(n^3)$ time per iteration!  You may
use a trivial predictor, or you may prefer to use an Euler predictor
based on a derivative computation like the one in Task 2.

\paragraph{Task 4}
Plot the solutions (both $x$ and $r$) as a function of $\epsilon$ and
comment on the qualitative behavior.  Could you give an intuition that
might explain what happens?

\subsection{Solution sensitivity}

Four parameters is too many for a complete exploration,
but we certainly can compute the sensitivities with respect
to four parameters at a point.

\paragraph{Task 5}
Write a routine that computes
not only the solution for a given set of parameters, but also
the derivatives of $x$ and $r$ with respect to the parameter
tuple $(\epsilon, \phi, \rho, \tau)$.  That is, we want an
$n \times 4$ matrix of derivatives of $x$ and a length 4
vector of derivatives of $r$ with respect to the parameters:
\begin{lstlisting}[frame=single]
function [dx,dr] = sensitivity(epsilon, phi, rho, tau, n)
\end{lstlisting}
For an illustrative set of parameter values, plot each derivative.
Can you give an intuition for what you see?

\section{Going beyond}

As is often the case, the model we have described has several
(possibly too many) parameters, and several somewhat ad hoc choices;
for example, we could have used a different form for our $\beta$
function.  The last task is an open-ended request to go further.  Any
reasonable effort will be given full credit (if you are unsure whether
your effort is reasonable, ask!).

\paragraph{Task 6}
Extend the analysis in this project in some way.  Possible examples
could be
\begin{itemize}
\item Change the form of $\beta$
\item Change $\alpha$ (e.g.~letting $\phi$ vary with $\sigma$).
\item Analyze different types of mutation matrices.
\item Do a parameter study over some interesting parameter region.
\item Analyze the stability of the dynamics beyond $\epsilon = 0$.
\item Simulate the ODEs using MATLAB to show the dynamics.
\item Do a survey of any related models in the literature.
\end{itemize}

\end{document}
