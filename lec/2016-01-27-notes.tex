\documentclass[12pt, leqno]{article}
\usepackage{amsfonts}
\usepackage{amsmath}
\usepackage{amssymb}
\usepackage{fancyhdr}
\usepackage{hyperref}
\usepackage{tikz}
\usepackage{pgfplots}
\usepackage{listings}

\newcommand{\bbR}{\mathbb{R}}
\newcommand{\bbC}{\mathbb{C}}
\newcommand{\calV}{\mathcal{V}}
\newcommand{\calW}{\mathcal{W}}
\newcommand{\ddiag}{\operatorname{diag}}
\newcommand{\fl}{\operatorname{fl}}
\newcommand{\macheps}{\epsilon_{\mathrm{mach}}}
\newcommand{\matlab}{\textsc{Matlab}}

\newcommand{\hdr}[2]{
  \pagestyle{fancy}
  \lhead{Bindel, Spring 2016}
  \rhead{Numerical Analysis (CS 4220)}
  \fancyfoot{}
  \begin{center}
    {\large{\bf Notes for #1}}
  \end{center}
  \lstset{language=matlab,columns=flexible}  
}


\begin{document}
\hdr{2016-01-27}

\section*{What are we about?}

Welcome to ``Numerical Analysis: Linear and Nonlinear Equations'' (CS
4220, CS 5223, and Math 4260).  This is part of a pair of courses
offered jointly between CS and math that provide an introduction to
scientific computing.  My own tongue-in-cheek summary of scientific
computing is that it is the art of solving problems of continuous
mathematics fast enough and accurately enough.  Of course, what
constitutes ``fast enough'' and ``accurately enough'' depends on
context, and learning to reason about that context is also part of the
class.

Because our survey is partitioned into two semesters, we do not cover
all the standard topics in a single semester.  In particular, this
class will (mostly) not cover interpolation and function
approximation, numerical differentiation and quadrature, or the
solution of ordinary and partial differential equations.  We will
focus instead on numerical linear algebra, nonlinear equation solving,
and optimization.  Broadly speaking, we will spend the first half of
the semester on {\em factorization} methods for linear algebra
problems, and the latter half on {\em iterative} methods for both
linear and nonlinear problems.  As currently planned, the schedule
also includes time for one special topic exercise on randomized
methods in linear algebra that I expect will bring together several of
the themes from the course.

\subsection*{Mathematics, Computation, Application}

Our focus will be the mathematical and computational structure of
numerical methods.  But we use numerical methods to solve
problems from applications, and a scientific computing class with
no applications is far less rich and interesting than it ought to be.
So we will, when possible, try to bring in application examples.

The majority of students in the class come from computer science.  We
also have students from various disciplines in the mathematical
sciences (math, applied math, statistics, operations research),
physical sciences (physics, applied physics, astronomy), engineering
disciplines (mechanical, civil, electrical, and chemical engineering),
and a few others (economics, undeclared).  This means that students
come to the class with different levels of background and interest in
a variety of application domains.  Because of the nature of the
enrollment, many of my examples will come from areas conventionally
associated with computer science and mathematics, but there will also
be the odd example from physics or engineering.  So if we dig into an
application problem and you get lost, don't worry -- I don't expect
you to know this already!  Also, ask questions, as there are bound to
be others the class who are equally confused.

\subsection*{Cross-cutting themes}

There are some themes that cut across topics in the syllabus,
and I expect we will touch on these themes frequently through
the semester.  These include:
\begin{itemize}
\item {\bf Knowing the answer in advance} -- It's dangerous to go into
  a computation with no idea what to expect.  The structure of the
  problem and the solution affect how we choose methods and how we
  evaluate success.  A qualitative analysis or ballpark estimate of
  solution behavior is usually the first step to intelligently
  applying a numerical method.
\item {\bf Pictures and plots} -- Careful pictures tell us a lot.
  Plot an approximate solution.  Are there unexpected oscillations or
  negative values, or crazy-looking behaviors near the domain of the
  soution?  Maybe you should investigate!  Similarly, plots of error
  with respect to a spatial variable or a step number often provide
  key insights into whether a method is working as desired.
\item {\bf Documentation, testing, and error checking} -- When we
  write numerical codes, the implied agreement between the author of
  the code and the user of the code is often more subtle than
  the agreements behind other software interfaces.  Call a sort
  routine, and it will sort your data in some specified time.  Call a
  linear solver, and it will solve your problem in an amount of time
  that depends on the problem structure and with a level of accuracy
  that depends on the problem characteristics.  This makes good
  software hygiene -- careful documentation, testing, error
  checking, and design for reproducibility -- both tricky and important!
\item {\bf Modularity and composability} -- When we compose numerical
  methods, we have to worry about error.  Even if you expect only
  to use numerical building blocks (and never build them yourself), it
  is important to understand the types of error and performance
  guarantees one can make and how they are useful in reasoning about
  large computational codes.
\item {\bf Problem formulation and choice of representation} -- Often, the
  same problem can be posed in many different ways.  Some suggest
  simple, efficient numerical methods.  Others are impossibly hard.
  The key difference between the two is often in how we represent the
  problem data and the thing we seek.
\item {\bf Numerical anti-patterns} -- Some operations, such as
  computing explicit inverses and determinants, are perfectly natural
  in symbolic mathematics but turn out to be terrible ideas in
  numerical computations.  We will point these out as we come across
  them.
\item {\bf Time and memory scalability} -- We often want to solve
  big problems, and it is important to understand before we start
  whether we think we can solve a problem on a laptop in a second or
  two or if we really need a month on a supercomputer.  This means
  we would like a rough estimate -- usually posed in terms of order notation
  -- of the time and memory complexity of different algorithms.
\item {\bf Blocking and building with high-performance blocks} --
  Building fast codes is hard.  As numerical problem solvers, we would
  like someone else to do much of this hard work so that we can focus
  on other things.  This means we need to understand the common
  building blocks, and a little bit about not only their complexity,
  but also why they are fast or slow on real machines.
\item {\bf Performance tradeoffs in iterations} -- Iterative methods
  produce a sequence of approximate solutions that (one hopes) get
  closer and closer to the right answer.  To choose iterations
  intelligently, we need to understand the tradeoffs between the
  time to compute an iteration, the progress that one can make, and
  the overall stability of an iterative procedure.
\item {\bf Convergence monitoring and stopping} -- One of the hardest
  parts of designing an iterative method is often deciding when to
  stop.  This point will recur several times in the second half of the
  semester.
\item {\bf Use of approximations and surrogates} -- Simple surrogate
  models are an important part of the design of nonlinear iterations.
  We will be particularly interested in local polynomial
  approximations, but we may talk about some others as well.
\end{itemize}

\section*{Logistics}

We will go through the syllabus in detail, but at a high level you
should plan on six homeworks (individual) and three projects (in
pairs), a midterm, and a final.  I will also ask you for feedback at
the middle and end of the semester, and this counts for credit.  I
will give ``problems of the day'' to help study, but we will not use
these directly to grade you.

Homework and projects are due via CMS by midnight on Fridays; we allow
some ``slip days'' so that you can work on an assignment through the
weekend if needed.  We have office hours scheduled before class on
Wednesday, 10-11 Thursday, and 10-12 on Friday.  You can also request
office hours by appointment.

\subsection*{Infrastructure}

Class notes and assignments, as well as class announcements, will be
posted on the course home page.  For submissions, solutions, and
grades, we will use the CS Course Management System (CMS) software.
For class discussion, we will use Piazza.  There are links from each
of these pages to the others; I recommend you use the class web page
as your starting point.

We will use MATLAB in our notes, but programming assignments may be
done in MATLAB (or Octave) or in Python.

The course web page is maintained from a repository on GitHub.
I encourage you to submit corrections or enhancements by pull
request!

\subsection*{Background}

The formal prerequisites for the class are linear algebra at the level
of Math 2210 or 2940 or equivalent and a CS 1 course in any language.
We also recommend one additional math course at the 3000 level or
above; this is essentially a proxy for ``sufficient mathematical
maturity.''

In practice: I will assume you know some multivariable calculus
and linear algebra, and that your CS background includes not only
basic programming but also some associated mathematical concepts
(e.g.~order notation and a little graph theory).  If you feel your
background is weak in these areas, please talk to us.

% CV 12 commandments
% ==================
% Matvec = linear combo of columns
% Inner product = sum of products
% Order of ops is important
% Matrix * diag = col scaling
% Diag * matrix = row scaling
% Never form explicit diag matrix
% Never form explicit rank 1
% Matrix * matrix = collection of matrix * vector
% Matrix * matrix = dot products
% Matrix * matrix = sum of rank one
% Matrix * matrix = linear combo of rows from second matrix
% Matrix * matrix = linear combo of cols from first matrix

% DSB
% ===
% Blocking
% Dynamic programming
% Sparsity and implicit multiply

\section*{Matrix algebra versus linear algebra}

\begin{enumerate}
\item
  Matrices are extremely useful.  So are linear transformations.  But note
  that matrices and linear transformations are {\em different} things!
  Matrices {\em represent} finite-dimensional linear transformations with
  respect to particular bases.  Change the bases, and you change the
  matrix, if not the underlying operator.  Much of the class will be about
  finding the right basis to make some property of the underlying transformation
  obvious, and about finding changes of basis that are ``nice'' for
  numerical work.

\item
  A linear transformation may correspond to different matrices depending on the
  choice of basis, but that doesn't mean the linear transformation is always the
  thing.  For some applications, the matrix itself has meaning, and the
  associated linear operator is secondary.  For example, if I look at 
  an adjacency matrix for a graph, I probably really do care about the matrix --
  not just the linear transformation.

\item
  Sometimes, we can apply a linear transformation even when we don't have an
  explicit matrix.  For example, suppose $F : \bbR^n \rightarrow
  \bbR^m$, and I want to compute 
  $\partial F / \partial v|_{x_0} = (\nabla F(x_0)) \cdot v$.
  Even without an explicit matrix for $\nabla F$, I can compute
  $\partial F / \partial v|_{x_0} \approx F(x_0 + hv)-F(x_0))/h$.
  There are many other linear transformations, too, for which it is
  more convenient to apply the transformations than to write down the
  matrix -- using the FFT for the Fourier transform operator, for
  example, or fast multipole methods for relating charges to potentials
  in an $n$-body electrostatic interaction.

\end{enumerate}


\section*{Matrix-vector multiply}

Let us start with a very simple \matlab\ program for matrix-vector
multiplication:
\begin{verbatim}
  function y = matvec1(A,x)
  % Form y = A*x (version 1)

  [m,n] = size(A);
  y = zeros(m,1);
  for i = 1:m
    for j = 1:n
      y(i) = y(i) + A(i,j)*x(j);
    end
  end
\end{verbatim}
We could just as well have switched the order of the $i$ and $j$ loops
to give us a column-oriented rather than row-oriented version of the algorithm.
Let's consider these two variants, written more compactly:
\begin{verbatim}
  function y = matvec2_row(A,x)
  % Form y = A*x (row-oriented)

  [m,n] = size(A);
  y = zeros(m,1);
  for i = 1:m
    y(i) = A(i,:)*x;
  end


  function y = matvec2_col(A,x)
  % Form y = A*x (column-oriented)

  [m,n] = size(A);
  y = zeros(m,1);
  for j = 1:n
    y = y + A(:,j)*x(j);
  end
\end{verbatim}

It's not too surprising that the builtin matrix-vector multiply routine in
\matlab\ runs faster than either of our {\tt matvec2} variants, but there
are some other surprises lurking.  Try timing each of these matrix-vector
multiply methods for random square matrices of size 4095, 4096, and 4097,
and see what happens.  Note that you will want to run each code many times
so that you don't get lots of measurement noise from finite timer granularity;
for example, try
\begin{verbatim}
  tic;          % Start timer
  for i = 1:100 % Do enough trials that it takes some time
    % ...         Run experiment here
  end
  toc           % Stop timer
\end{verbatim}

% Matrix multiplication and blocking
% Memory access and vectorization issues; BLAS routines
% Matrix representation

\section*{Basic matrix-matrix multiply}

The classic algorithm to compute $C := C + AB$ is
\begin{verbatim}
  for i = 1:m
    for j = 1:n
      for k = 1:p
        C(i,j) = C(i,j) + A(i,k)*B(k,j);
      end
    end
  end
\end{verbatim}
This is sometimes called an {\em inner product} variant of
the algorithm, because the innermost loop is computing a dot
product between a row of $A$ and a column of $B$.  We can
express this concisely in MATLAB as
\begin{verbatim}
  for i = 1:m
    for j = 1:n
      C(i,j) = C(i,j) + A(i,:)*B(:,j);
    end
  end
\end{verbatim}
There are also {\em outer product} variants of the algorithm
that put the loop over the index $k$ on the outside, and thus
computing $C$ in terms of a sum of outer products:
\begin{verbatim}
  for k = 1:p
    C = C + A(:,k)*B(k,:);
  end
\end{verbatim}

\section*{Blocking and performance}

The basic matrix multiply outlined in the previous section will
usually be at least an order of magnitude slower than a well-tuned
matrix multiplication routine.  There are several reasons for this
lack of performance, but one of the most important is that the basic
algorithm makes poor use of the {\em cache}.
Modern chips can perform floating point arithmetic operations much
more quickly than they can fetch data from memory; and the way that
the basic algorithm is organized, we spend most of our time reading
from memory rather than actually doing useful computations.
Caches are organized to take advantage of {\em spatial locality},
or use of adjacent memory locations in a short period of program execution;
and {\em temporal locality}, or re-use of the same memory location in a
short period of program execution.  The basic matrix multiply organizations
don't do well with either of these.
A better organization would let us move some data into the cache
and then do a lot of arithmetic with that data.  The key idea behind
this better organization is {\em blocking}.

When we looked at the inner product and outer product organizations
in the previous sections, we really were thinking about partitioning
$A$ and $B$ into rows and columns, respectively.  For the inner product
algorithm, we wrote $A$ in terms of rows and $B$ in terms of columns
\[
  \begin{bmatrix} a_{1,:} \\ a_{2,:} \\ \vdots \\ a_{m,:} \end{bmatrix}
  \begin{bmatrix} b_{:,1} & b_{:,2} & \cdots & b_{:,n} \end{bmatrix},
\]
and for the outer product algorithm, we wrote $A$ in terms of colums
and $B$ in terms of rows
\[
  \begin{bmatrix} a_{:,1} & a_{:,2} & \cdots & a_{:,p} \end{bmatrix}
  \begin{bmatrix} b_{1,:} \\ b_{2,:} \\ \vdots \\ b_{p,:} \end{bmatrix}.
\]
More generally, though, we can think of writing $A$ and $B$ as 
{\em block matrices}: 
\begin{align*}
  A &=
  \begin{bmatrix}
    A_{11} & A_{12} & \ldots & A_{1,p_b} \\
    A_{21} & A_{22} & \ldots & A_{2,p_b} \\
    \vdots & \vdots &       & \vdots \\
    A_{m_b,1} & A_{m_b,2} & \ldots & A_{m_b,p_b}
  \end{bmatrix} \\
  B &=
  \begin{bmatrix}
    B_{11} & B_{12} & \ldots & B_{1,p_b} \\
    B_{21} & B_{22} & \ldots & B_{2,p_b} \\
    \vdots & \vdots &       & \vdots \\
    B_{p_b,1} & B_{p_b,2} & \ldots & B_{p_b,n_b}
  \end{bmatrix} 
\end{align*}
where the matrices $A_{ij}$ and $B_{jk}$ are compatible for matrix
multiplication.  Then we we can write the submatrices of $C$ in terms
of the submatrices of $A$ and $B$
\[
  C_{ij} = \sum_k A_{ij} B_{jk}.
\]

\section*{The lazy man's approach to performance}

An algorithm like matrix multiplication seems simple, but there is a
lot under the hood of a tuned implementation, much of which has to do
with the organization of memory.  We often get the best ``bang for our
buck'' by taking the time to formulate our algorithms in block terms,
so that we can spend most of our computation inside someone else's
well-tuned matrix multiply routine (or something similar).  There are
several implementations of the Basic Linear Algebra Subroutines
(BLAS), including some implementations provided by hardware vendors
and some automatically generated by tools like ATLAS.  The best BLAS
library varies from platform to platform, but by using a good BLAS
library and writing routines that spend a lot of time in {\em level 3}
BLAS operations (operations that perform $O(n^3)$ computation on
$O(n^2)$ data and can thus potentially get good cache re-use), we can
hope to build linear algebra codes that get good performance across
many platforms.

This is also a good reason to use \matlab: it uses pretty good BLAS libraries,
and so you can often get surprisingly good performance from it for the types
of linear algebraic computations we will pursue.

\section*{Problems to ponder}

Unless otherwise stated, assume $A, B \in \bbR^{n \times n}$
(square real $n \times n$ matrices), $u, v, x, y$ are vectors in
$\bbR^n$, and $D = \operatorname{diag}(d)$ is a diagonal $n \times n$.
\begin{enumerate}
\item
  Describe the effect of pre- and post-multiplying $A$ by $D$;
  that is, what are $DA$ and $AD$?
\item
  How many floating point operations are needed to evaluate the
  following (assuming ordinary order of operations)?
  \begin{enumerate}
  \item $(uv^T) A$
  \item $u (v^T A)$
  \item $A (u v^T) B$
  \item $(A u) (v^T V)$
  \item $A D x$
  \item $A (Dx)$
  \end{enumerate}
\item
  Describe a brief snippet of MATLAB code to form the most efficient
  versions of the above expressions.
\item
  The standard tridiagonal matrix $T_N \in \bbR^{N \times N}$ acts
  on the vector $u$ in the following way:
  \[
    (Tu)_i = -u_{i-1} + 2u_i - u_{i+1}
  \]
  with the convention $u_0 = u_{N+1} = 0$.
  \begin{enumerate}
  \item What is $T_5$, written explicitly?
  \item Write a MATLAB snippet to evaluate $Tu$ in $O(N)$ time.
  \end{enumerate}
\item
  Let $E \in \bbR^{n \times n}$ be the matrix of all ones.
  Describe an $O(n)$ approach to compute $Ev$.
\item
  The operation $\operatorname{triu}(E)$ takes the upper triangular
  part of $E$; for example, for $n = 3$, we have
  \[
  \operatorname{triu}(E) =
  \begin{bmatrix}
    1 & 1 & 1 \\
    0 & 1 & 1 \\
    0 & 0 & 1 
  \end{bmatrix}
  \]
  In general, describe an $O(n)$ approach to compute
  $\operatorname{triu}(E) v$.
\end{enumerate}

\end{document}
