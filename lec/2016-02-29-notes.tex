\documentclass[12pt, leqno]{article}
\usepackage{amsfonts}
\usepackage{amsmath}
\usepackage{amssymb}
\usepackage{fancyhdr}
\usepackage{hyperref}
\usepackage{tikz}
\usepackage{pgfplots}
\usepackage{listings}

\newcommand{\bbR}{\mathbb{R}}
\newcommand{\bbC}{\mathbb{C}}
\newcommand{\calV}{\mathcal{V}}
\newcommand{\calW}{\mathcal{W}}
\newcommand{\ddiag}{\operatorname{diag}}
\newcommand{\fl}{\operatorname{fl}}
\newcommand{\macheps}{\epsilon_{\mathrm{mach}}}
\newcommand{\matlab}{\textsc{Matlab}}

\newcommand{\hdr}[2]{
  \pagestyle{fancy}
  \lhead{Bindel, Spring 2016}
  \rhead{Numerical Analysis (CS 4220)}
  \fancyfoot{}
  \begin{center}
    {\large{\bf Notes for #1}}
  \end{center}
  \lstset{language=matlab,columns=flexible}  
}


\begin{document}
\hdr{2016-02-29}

% Q-less QR
% Constrained problems
% Weighted least squares
% IRLS

\section*{Sparse least squares and Q-less QR}

Suppose we want to solve a full-rank least squares problem in which $A$
is large and sparse.  In principle, we could solve the problem via the
normal equations
\[
  A^T A x = A^T b,
\]
or introduce $A = QR$ and multiply $A^T A x = R^T R x = b$ by $R^{-T}$ to find
\[
  R x = R^{-T} A^T b = Q^T b.
\]
Note that there is a very close relation between these approaches; the
matrix $R$ in the QR decomposition is a Cholesky factor of $A^T A$ in
the normal equations (possibly scaled by a diagonal matrix with
$\pm 1$ on the diagonal).  As we have discussed, it may not be advantageous to 
use the normal equations directly, since forming and factoring $A^T A$
brings in the square of the condition number.  On the other hand,
in a sparse setting it's not necessarily such a good idea to use the
usual QR approach, since even if $A$ and $A^TA$ are sparse, $Q$ in general
will not be.  Consequently, when $A$ is sparse, we would typically use the
following steps\footnote{%
In MATLAB, you can also use backslash
to solve a least squares problem, and it will do the right thing if
$A$ is sparse.}
\begin{enumerate}
\item
  Possibly permute the columns of $A$ so that the Cholesky factor of $A^T A$
  (or the factor $R$, which has the same structure) remains sparse.
  See {\tt help colamd} for an example of a generally good permutation.
\item
  Compute a ``Q-less'' QR decomposition, e.g. {\tt R = qr(A,0)} in \matlab
  where $A$ is sparse.  This does not compute the (usually very dense) $Q$
  factor explicitly.  It also does not form $A^T A$ explicitly.  If
  the right hand side $b$ is known initially, the \matlab\ {\tt qr}
  function can compute $Q^T b$ implicitly at the same time it does the
  QR factorization.
\item
  Compute $Q^T b$ as $R^{-T} (A^T b)$, since the latter computation involves
  only a sparse multiply and a sparse triangular solve.
\item
  Solve $R x = Q^T b$.
\end{enumerate}
It's not a bad idea to do iterative refinement after this --- see {\tt help qr}
in MATLAB:
\begin{lstlisting}
  x = R\(R'\(A'*b))
  r = b-A*x;
  e = R\(R'\(A'*r));
  x = x + e;
\end{lstlisting}

Of course, just as some square sparse systems cannot be re-ordered so
that the LU factorization will be sparse, some sparse least squares
problems cannot be re-ordered so that the QR factorization will be
sparse.  Sometimes, these problems can be more effectively treated by
iterative methods of the sort we will discuss later in the class.

\section*{Weight patiently}

So far, we have considered least squares problems involving the
standard Euclidean norm in $\bbR^n$.  But while we clearly take
advantage of Euclidean structure in thinking about least squares,
nothing requires that we look to the {\em standard} inner product
for that structure.  For example, consider the problem
\[
  \mbox{minimize } \|Ax-b\|_M^2
\]
where $M$ is a general symmetric and positive definite matrix.
We can approach this problem through a generalization of the normal
equations, e.g.
\[
  A^T M A x = A^T M b.
\]
Alternately, we can observe that if $M = R_M^T R_M$ is a Cholesky
factorization, then the $M$-norm least squares problem is
equivalent to a least squares problem with respect to the standard
inner product:
\[
  \mbox{minimize } \|R_M (Ax-b)\|^2.
\]

Frequently, we care about {\em weighted} least squares problems
\[
  \mbox{minimize } \sum_{i=1}^m w_i^2 r_i^2
\]
where $w_i^2$ are a set of weights.  This may be appropriate if some
rows represent measures that are more certain than others; for
example, we may try to identify and down-weight outliers that do not
fit the general trend of the rest of the data.  Alternately, we may
incorporate weights if different rows are naturally scaled
differently.  A weighted least squares problem can be re-phrased as a
standard least squares problem:
\[
  \mbox{minimize } \| D^{-1} (Ax - b) \|^2
\]
where $D = \operatorname{diag}(w_1, \ldots, w_m)$ is a diagonal
scaling matrix corresponding to the weights.

Weighted least squares makes a good building block for solving
more general fitting problems.
The {\em iteratively reweighted least squares} procedure (IRLS)
is used to solve problems of the form
\[
  \mbox{minimize } \sum_{i=1}^m l(r_i)
\]
where $l$ is some positive-definite loss function.  If we let
$l(r) = r^2$, we get the standard least squares problem; but
standard least squares problems are sensitive to outliers,
and may not be appropriate when measurement errors come from
a non-Gaussian distribution with heavy tails.  In some procedures,
one tries to identify outliers statically, then downweights them
in an ordinary least squares problem.  In other procedures,
one uses a loss function $l$ that grows less quickly as a function
of the argument; for example, $l(r) = |r|$ corresponds to
{\em least absolute deviation} fitting.  A popular choice due to
Huber combines the least squares and least absolute deviation loss:
\[
l(r) = \begin{cases}
  \frac{r^2}{2\epsilon}, & |r| \leq \epsilon \\
  |r|-\frac{\epsilon}{2}, & |r| > \epsilon.
  \end{cases}
\]
There are other choices as well, such as the Tukey biweight function.
The idea in IRLS is to approximate the solution to the general loss
minimization throguh a sequence of least square problems; at each
step, weights $w_j = l'(\hat{r}_j/s)/\hat{r}_j$, where $s$ is a scale
parameter and $\hat{r}$ is the residual vector from the previous step.
We will return to IRLS later in the class when we talk about
iterations for nonlinear least squares problems.

\section*{Consider constraint}

Weighted least squares is a good way to construct fits in which some
equations are more important than others.  But what if we want to
enforce some equations {\em exactly}?  That is, we consider the
{\em linearly constrained least squares problem}
\[
  \mbox{minimize } \|Ax-b\|^2 \mbox{ s.t. } Bx = c
\]
where $B \in \bbR^{l \times n}$ with $l < n$.

There are several approaches to solving this problem.  One in
particular relates to the picture we saw when discussing the QR
approach to solving least squares problems.  We start by finding some
solution $y$ to the constraint equation $By = c$, and form a
{\em homogeneous} problem in terms of a correction $z = x-y$
\[
  \mbox{minimize } \|Az-s\|^2 \mbox{ s.t. } Bz = 0, \quad s = b-Ay.
\]
In the usual QR approach, we consider an orthogonal transformation to
the residual space such that part of the residual is independent of
$x$ and part of the residual can be set to zero.  In the constrained
setting, we also want to transform the space where the variable
lives.  That is, suppose we write
\[
  B^T = \begin{bmatrix} U_1 & U_2 \end{bmatrix} \begin{bmatrix} R_B \\ 0 \end{bmatrix}
\]
Then the condition $Bz = 0$ is equivalent to the condition that
$z = U_1 w$.  Therefore, we want to solve the problem
\[
  \mbox{minimize } \|A U_1 w-s\|^2.
\]

This method of handling constraints is fine in the dense case.  In the
sparse case, we might prefer an alternate approach based on the
method of {\em Lagrange multipliers}, which we will return to in
more detail later in the semester.  In this method, we want to find
a stationary point for the {\em augmented Lagrangian}
\[
  L(x,\lambda) = \frac{1}{2} \|Ax-b\|^2 + \lambda^T (Bx-c).
\]
Differentiating with respect to $x$ and $\lambda$, we have
\[
\begin{bmatrix} \delta x \\ \delta \lambda \end{bmatrix}^T \\
\begin{bmatrix}
  A^T A x + B^T \lambda - A^T b \\
  Bx - c
\end{bmatrix} = 0,
\]
i.e.
\[
  \begin{bmatrix} A^T A & B^T \\ B & 0 \end{bmatrix}
  \begin{bmatrix} x \\ \lambda \end{bmatrix} =
  \begin{bmatrix} A^T b \\ c \end{bmatrix}.
\]
Applying block Gaussian elimination, we have
\[
  B (A^T A)^{-1} B^T \lambda = B (A^T A)^{-1} A^T b - c.
\]
Using the economy factorization $A = QR$, and introducing
$M = B R^{-1}$, we have
\[
  (M^T M) \lambda = M Q^T b - c.
\]
From there, we can solve for $\lambda$ and then do some algebra,
which leads to the following MATLAB code:
\begin{lstlisting}
  % Factorizations independent of RHS
  [Q,R] = qr(A, 0);
  M = B/R;
  [Q2,T] = qr(M,0);

  % Part that depends on the RHS
  x0 = R\(Q'*b);           % Unconstrained solve
  lambda = T\(T'\(B*x0-c));  % Compute multipliers
  x = x0-R\(M'*lambda);    % Correction to unconstrained solve
\end{lstlisting}

\newpage
\section*{Problems to ponder}

\begin{enumerate}
\item Derive the generalized normal equations for minimizing
  $\|Ax-b\|_M^2$.
\item Write a MATLAB code to solve the linearly constrained least
  squares problem using the first approach described (QR decomposition
  of the constraint matrix).
\item Complete the algebra to get from the extended system
  formulation for the linearly constrained least squares problem
  to the MATLAB code at the end of the notes.
\end{enumerate}

\end{document}

