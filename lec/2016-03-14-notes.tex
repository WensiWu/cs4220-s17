\documentclass[12pt, leqno]{article} %% use to set typesize 
\usepackage{amsfonts}
\usepackage{amsmath}
\usepackage{amssymb}
\usepackage{fancyhdr}
\usepackage{hyperref}
\usepackage{tikz}
\usepackage{pgfplots}
\usepackage{listings}

\newcommand{\bbR}{\mathbb{R}}
\newcommand{\bbC}{\mathbb{C}}
\newcommand{\calV}{\mathcal{V}}
\newcommand{\calW}{\mathcal{W}}
\newcommand{\ddiag}{\operatorname{diag}}
\newcommand{\fl}{\operatorname{fl}}
\newcommand{\macheps}{\epsilon_{\mathrm{mach}}}
\newcommand{\matlab}{\textsc{Matlab}}

\newcommand{\hdr}[2]{
  \pagestyle{fancy}
  \lhead{Bindel, Spring 2016}
  \rhead{Numerical Analysis (CS 4220)}
  \fancyfoot{}
  \begin{center}
    {\large{\bf Notes for #1}}
  \end{center}
  \lstset{language=matlab,columns=flexible}  
}


\begin{document}
\hdr{2016-03-14}

\section*{Overview}

After this week (and the associated problems), you should come away
with some understanding of
\begin{itemize}
\item {\em Algorithms} for equation solving, particularly bisection,
  Newton, secant, and fixed point iterations.
\item {\em Analysis} of error recurrences in order fo find rates of
  convergence for algorithms; you should also understand a little
  about analyzing the sensitivity of the root-finding problem itself.
\item {\em Application} of standard root-finding procedures to real
  problems.  This frequently means some sketches and analysis done in
  advance in order to figure out appropriate rescalings and changes of
  variables, handle singularities, and find good initial guesses (for
  Newton) or bracketing intervals (for bisection).
\end{itemize}


\section*{A little long division}

Let's begin with a question: Suppose I have a machine with hardware
support for addition, subtraction, multiplication, and scaling by
integer powers of two (positive or negative).  How can I implement
reciprocation?  That is, if $d > 1$ is an integer, how can I compute
$1/d$ without using division?

This is a linear problem, but as we will see, it presents many of
the same issues as nonlinear problems.

\subsection*{Method 1: From long division to bisection}

\begin{figure}
\lstinputlisting{2016-03-14-codes/reciprocal_divide.m}
\caption{Approximate $1/d$ by $n$ steps of binary long division.}
\label{fig-division}
\end{figure}

\begin{figure}
\lstinputlisting{2016-03-14-codes/reciprocal_bisect.m}
\caption{Approximate $1/d$ by $n$ steps of bisection.}
\label{fig-bisect}
\end{figure}

Maybe the most obvious algorithm to compute $1/d$ is binary long
division (the binary version of the decimal long division that
we learned in grade school).  To compute a bit in the $k$th place
after the binary point (corresponding to the value $2^{-k}$), 
we see whether $2^{-k} d$ is greater than the current remainder;
if it is, then we set the bit to one and update the remainder.
This algorithm is shown in Figure~\ref{fig-division}.

At step $k$ of long division, we have an approximation $\hat{x}$,
$\hat{x} \leq 1/d < \hat{x}+2^{-k}$, and a remainder $r = 1-d
\hat{x}$.  Based on the remainder, we either get a zero bit (and
discover $\hat{x} \leq 1/d < \hat{x}+2^{-(k+1)}$), or we get a one bit
(i.e. $\hat{x}+2^{-(k+1)} \leq 1/d < \hat{x}+2^{-k}$).  That is,
the long division algorithm is implicitly computing interals that
contain $1/d$, and each step cuts the interval size by a factor of
two.  This is characteristic of {\em bisection}, which finds a zero
of any continuous function $f(x)$ by starting with a bracketing interval
and repeatedly cutting those intervals in half.  We show the
bisection algorithm in Figure~\ref{fig-bisect}.

\subsection*{Method 2: Almost Newton}

You might recall {\em Newton's method} from a calculus class.
If we want to estimate a zero near $x_k$, we take the first-order
Taylor expansion near $x_k$ and set that equal to zero:
\[
  f(x_{k+1}) \approx f(x_k) + f'(x_k)(x_{k+1}-x_k) = 0.
\]
With a little algebra, we have
\[
  x_{k+1} = x_k - f'(x_k)^{-1} f(x_k).
\]
Note that if $x_*$ is the actual root we seek, then
Taylor's formula with remainder yields
\[
  0 = f(x_*) = f(x_k) + f'(x_k)(x_*-x_k) + \frac{1}{2} f''(\xi) (x_*-x_k)^2.
\]
Now subtract the Taylor expansions for $f(x_{k+1})$ and $f(x_*)$ to get
\[
  f'(x_k)(x_{k+1}-x_*) + \frac{1}{2} f''(\xi) (x_k-x_*)^2 = 0.
\]
This gives us an iteration for the error $e_k = x_k-x_*$:
\[
  e_{k+1} = -\frac{1}{2} \frac{f''(\xi)}{f'(x_k)} e_k^2.
\]
Assuming that we can bound $f''(\xi)/f(x_k)$ by some modest constant $C$,
this implies that a small error at $e_k$ leads to a {\em really}
small error $|e_{k+1}| \leq C|e_k|^2$ at the next step.  This behavior,
where the error is squared at each step, is {\em quadratic convergence}.

If we apply Newton iteration to $f(x) = dx-1$, we get
\[
  x_{k+1} = x_k - \frac{dx_k-1}{d} = \frac{1}{d}.
\]
That is, the iteration converges in one step.  But remember that
we wanted to avoid division by $d$!  This is actually not uncommon:
often it is inconvenient to work with $f'(x_k)$, and so we instead
cook up some approximation.  In this case, let's suppose we have
some $\hat{d}$ that is an integer power of two close to $d$.
Then we can write a modified Newton iteration
\[
  x_{k+1} = x_k - \frac{dx_k-1}{\hat{d}}
         = \left(1-\frac{d}{\hat{d}} \right) x_k + \frac{1}{\hat{d}}.
\]
Note that $1/d$ is a {\em fixed point} of this iteration:
\[
  \frac{1}{d} 
         = \left(1-\frac{d}{\hat{d}} \right) \frac{1}{d} + \frac{1}{\hat{d}}.
\]
If we subtract the fixed point equation from the iteration equation,
we have an iteration for the error $e_k = x_k-1/d$:
\[
   e_{k+1} = \left(1-\frac{d}{\hat{d}} \right) e_k.
\]
So if $|d-\hat{d}|/|d| < 1$, the errors will eventually go to zero.
For example, if we choose $\hat{d}$ to be the next integer power of
two larger than $d$, then $|d-\hat{d}|/|\hat{d}| < 1/2$, and we get at
least one additional binary digit of accuracy at each step.

When we plot the error in long division, bisection, or our modified
Newton iteration on a semi-logarithmic scale, the decay in the error
looks (roughly) like a straight line.  That is, we have {\em linear}
convergence.  But we can do better!

\subsection*{Method 3: Actually Newton}

We may have given up on Newton iteration too easily.  In many problems,
there are multiple ways to write the same nonlinear equation.  For example,
we can write the reciprocal of $d$ as $x$ such that $f(x) = dx-1 = 0$, or
we can write it as $x$ such that $g(x) = x^{-1} - d = 0$.  If we apply
Newton iteration to $g$, we have
\[
  x_{k+1} = x_k - \frac{g(x_k)}{g'(x_k)} 
         = x_k + x_k^{2} (x_k^{-1}-d)
         = x_k (2-dx_k).
\]
As before, note that $1/d$ is a fixed point of this iteration:
\[
  \frac{1}{d} = \frac{1}{d} \left( 2 - d \frac{1}{d} \right).
\]
Given that $1/d$ is a fixed point, we have some hope that this iteration
will converge --- but when, and how quickly?  To answer these questions,
we need to analyze a recurrence for the error.

We can get a recurrence for error by subtracting
the true answer $1/d$ from both sides of the iteration equation
and doing some algebra:
\begin{align*}
  e_{k+1} &= x_{k+1}-d^{-1} \\
         &= x_k(2-dx_k)-d^{-1} \\
         &= -d (x_k^2 - 2d^{-1} x_k + d^{-2}) \\
         &= -d (x_k-d^{-1})^2 \\
         &= -d e_k^2
\end{align*}
In terms of the relative error $\delta_k = e_k/d^{-1} = d e_k$,
we have
\[
  \delta_{k+1} = -\delta_{k}^2.
\]
If $|\delta_0| < 1$, then this iteration converges --- and once
convergence really sets in, it is ferocious, roughly doubling the
number of correct digits at each step.  Of course, if $|\delta_0| > 1$,
then the iteration diverges with equal ferocity.  Fortunately,
we can get a good initial guess in the same way we got a good guess
for the modified Newton iteration: choose the first guess to be
a nearby integer power of two.

On some machines, this sort of Newton iteration (intelligently
started) is actually the preferred method for division.

\section*{The big picture}

% General problem + assumptions (continuity? differentiability?)
% Nonlinear equations -- can't get it exactly even in principle!

% In general: need an initial guess (or bracketing interval)
% The simplest approach: bisection (problem: 1D!  Note on poles.)
% Newton iteration: basic concept
% Quadratic convergence
% Approximating Newton 1: secant iteration
% Approximating Newton 2: approximate Jacobian

% Some examples:
% - Bisection on tan(x) to get a pole
% - Fast Newton iteration for division (x = x*(2-dx))
%   Note: when does this converge?
% - Noisy function values due to Horner
% - Square root and convergence from above
% - Square root and linear convergence to a multiple root
% - Some nasty example from RCMR?

Let's summarize what we have learned from this example 
(and generalize slightly to the case of solving $f(x) = 0$
for more interesting $f$):
\begin{itemize}
\item
  {\em Bisection} is a general, robust strategy.  We just need that
  $f$ is continuous, and that there is some interval $[a,b]$ so that
  $f(a)$ and $f(b)$ have different signs.  On the other hand, it is
  not always easy to get a bracketing interval; and once we do,
  bisection only halves that interval at each step, so it may take
  many steps to reach an acceptable answer.  Also, bisection is an
  intrinsically one-dimensional construction.
\item
  {\em Newton iteration} is a standard workhorse based on
  finding zeros of successive linear approximations to $f$.
  When it converges, it converges ferociously quickly.
  But Newton iteration requires that we have a derivative
  (which is sometimes inconvient), and we may require a good
  initial guess.
\item
  A {\em modified Newton iteration} sometimes works well
  if computing a derivative is a pain.  There are many ways
  we can modify Newton method for our convenience; for example,
  we might choose to approximate $f'(x_k)$ by some fixed value,
  or we might use a secant approximation.
\item
  It is often convenient to work with {\em fixed point iterations} of the form
  \[
    x_{k+1} = g(x_k),
  \]
  where the number we seek is a fixed point of $g$ ($x_* = g(x_*)$).
  Newton-like methods are an example of fixed point iteration, but there
  are others.  Whenever we have a fixed point iteration, we can try
  to write an iteration for the error:
  \[
    e_{k+1} = x_{k+1}-x_* = g(x_k)-g(x_*) = g(x_*+e_k)-g(x_*).
  \]
  How easy it is to analyze this error recurrence depends somewhat on
  the properties of $g$.  If $g$ has two derivatives, we can write
  \[
    e_{k+1} = g'(x_*) e_k + \frac{1}{2} g''(\xi_k) e_k^2 
               \approx g'(x_*) e_k.
  \]
  If $g'(x_*) = 0$, the iteration converges {\em superlinearly}.
  If $0 < |g'(x_*)| < 1$, the iteration converges linearly, 
  and $|g'(x_*)|$ is the rate constant.  If $|g'(x_*)| > 1$, the
  iteration diverges.
\end{itemize}

\end{document}
