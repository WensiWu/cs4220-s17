\documentclass[12pt, leqno]{article} %% use to set typesize 
\usepackage{amsfonts}
\usepackage{amsmath}
\usepackage{amssymb}
\usepackage{fancyhdr}
\usepackage{hyperref}
\usepackage{tikz}
\usepackage{pgfplots}
\usepackage{listings}

\newcommand{\bbR}{\mathbb{R}}
\newcommand{\bbC}{\mathbb{C}}
\newcommand{\calV}{\mathcal{V}}
\newcommand{\calW}{\mathcal{W}}
\newcommand{\ddiag}{\operatorname{diag}}
\newcommand{\fl}{\operatorname{fl}}
\newcommand{\macheps}{\epsilon_{\mathrm{mach}}}
\newcommand{\matlab}{\textsc{Matlab}}

\newcommand{\hdr}[2]{
  \pagestyle{fancy}
  \lhead{Bindel, Spring 2016}
  \rhead{Numerical Analysis (CS 4220)}
  \fancyfoot{}
  \begin{center}
    {\large{\bf Notes for #1}}
  \end{center}
  \lstset{language=matlab,columns=flexible}  
}


\begin{document}
\hdr{2016-03-16}

\section*{The best of all possible worlds}

Last time, we discussed three methods of solving $f(x) = 0$:
Newton, modified Newton, and bisection.  Newton is potentially faster
than bisection; bisection is more reliable.  Ideally, we would like
something that is both fast {\em and} robust.  We consider two
different approaches to this problem: Brent's method (1D), and
globalized Newton (which will generalize).

\subsection*{Newton with line search}

One difficulty with Newton iteration is that sometimes the step is
in the right direction, but has the wrong magnitude.  We can get
around this with a {\em line search} strategy: we propose a Newton
step, but accept the step only if it reduces $|f(x)|$.  If $|f(x)|$
goes up, we go in the same direction but by a smaller amount;
a typical choice is to cut the step in half.  In code, we have the following.

\lstinputlisting{2016-03-16-codes/simple_newton.m}

Newton with line search converges more frequently than Newton with no
guards, but it can still go astray.  There is nothing in the setup for
a guarded Newton iteration that guarantees the function $f$ even has a
zero; and even if it does, it is possible to set up functions where
Newton always heads in the wrong direction.  In order to go from
``works better than Newton'' to ``works all the time,'' we need
another trick.

\subsection*{Secant iteration and Brent's method}

One of the annoying properties of Newton's method is that it requires
that we compute the derivative of $f$.  In some cases, we may not have
this derivative in closed form, but we can always estimate using
finite differences:
\[
  f'(x) \approx \frac{f(x+h)-f(x)}{h}.
\]
In the setting of a root finding iteration, it is natural to use a
derivative approximation based on the last two steps of the iteration;
this gives us the {\em secant iteration}
\[
  x_{k+1} = x_k - \frac{f(x_k)(x_k-x_{k-1})}{f(x_{k})-f(x_{k-1})}.
\]
The secant iteration is superlinearly convergent, though not
quadratically convergent\footnote{You can read the convergence
  theory elsewhere.}
Unlike Newton, we need {\em two} starting points for the iteration;
but if we start with an interval $[a,b]$ such that $f$ has a sign
change, it is natural to choose $a$ and $b$ as the initial guesses.

Unfortunately, secant iteration can also go astray.  Fortunately, we
can combine secant iteration with bisection to get both speed and
robustness.  The basic idea is:
\begin{itemize}
\item
  At each step, maintain an interval $[\alpha,\beta]$ such that $f$
  has a sign change between the end points.
\item
  If secant iteration is converging quickly, try taking a new point
  based on a secant step.
\item
  If the secant step falls out of bounds, or if secant iteration has
  not improved the bounding interval sufficiently in the past few
  steps, consider a new point based on a bisection step.
\item
  Reduce the interval based on the sign of $f$ at the new point
  (whether from a bisection or a secant point) and repeat.
\end{itemize}
This combined method is known as {\em Brent's method}, and it is the
usual default root-finder for one-dimensional problems.
Unfortunately, it is an intrinsically one-dimensional process ---
there is no natural generalization with similar robustness properties
for solving systems of equations.

\section*{Use a routine, or roll your own?}

% Calling sequence for MATLAB's fzero.

The \matlab\ function {\tt fzero} is a fast, reliable black-box root-finding
algorithm based on a combination of bisection (for safety) and
interpolation-based methods (for speed).  If you provide an initial
interval containing exactly one zero, and if the root you seek is not
too sensitive, {\tt fzero} will find the root you seek to high accuracy
(the default relative error tolerance is about 2$\macheps$).  I use
the function often, and recommend it to you.

That said, there are a few reasons to write your own root-finding
algorithms, at least some of the time:
\begin{enumerate}
\item
  Not all the world is \matlab, and you may sometimes find that you
  have to write these things yourself.
\item 
  Black box approaches are far less useful for problems involving
  multiple variables.  Consequently, it's worth learning to write
  Newton-like methods in one variable so that you can learn their
  properties well enough to work with similar algorithms in more than
  one variable.
\item
  Actually walking through the internals of a root-finding algorithm
  can be a terrific way to gain insight into how to formulate your
  problems so that a standard root finder can solve them.
\end{enumerate}

\section*{Sensitivity and error}

% Condition number of a root.  Effects of roundoff.  Q: is the root the desired thing?
% Absolute and relative error tolerances.  Effects of conditioning.
% Scaling and dimensionality.  Definition of success.

Suppose we want to find $x_*$ such that $f(x_*) = 0$.  On the
computer, we actually have $\hat{f}(\hat{x}_*) = 0$.  We'll assume
that we're using a nice, robust code like {\tt fzero}, so we have a
very accurate zero of $\hat{f}$.  But this still leaves the question:
how well do $\hat{x}_*$ and $x_*$ approximate each other?
In other words, we want to know the sensitivity of the root-finding
problem.

If $\hat{x}_* \approx x_*$, then 
\[
  f(\hat{x}_*) \approx f'(x_*) (\hat{x}_*-x_*).
\]
Using the fact that $\hat{f}(\hat{x}_*) = 0$, we have that if
$|\hat{f}-f| < \delta$ for arguments near $x_*$, then
\[
  |f'(x_*) (\hat{x}_*-x_*)| \lesssim \delta.
\]
This in turn gives us
\[
  |\hat{x}_*-x_*| \lesssim \frac{\delta}{f'(x_*)}.
\]
Thus, if $f'(x_*)$ is close to zero, small rounding errors in the
evaluation of $f$ may lead to large errors in the computed root.

It's worth noting that if $f'(x_*) = 0$ (i.e. if $x_*$ is a multiple
root), that doesn't mean that $x_*$ is completely untrustworthy.  It
just means that we need to take more terms in a Taylor series in order
to understand the local behavior.  In the case $f'(x_*) = 0$, we have
\[
  f(\hat{x}_*) \approx \frac{1}{2} f''(x_*) (\hat{x}_*-\hat{x}_*),
\]
and so we have
\[
  |\hat{x}_*-\hat{x}_*| \leq \sqrt{\frac{2\delta}{f''(x_*)}}.
\]
So if the second derivative is well behaved and $\delta$ is on the
order of around $10^{-16}$, for example, our computed $\hat{x}$ might
be accurate to within an absolute error of around $10^{-8}$.

Understanding the sensitivity of root finding is not only important so
that we can be appropriately grim when someone asks for impossible
accuracy.  It's also important because it helps us choose problem
formulations for which it is (relatively) easy to get good accuracy.

\section*{Choice of functions and variables}

% Example: f(x) = sin(x)-12*x*(0.1-x)
% Alternative: g(x) = sinc(x/pi) - 12*(0.1-x).  Notice that the book
%   actually gets the definition of sinc slightly wrong...
% Ideally, avoid choosing something long and shallow

% General trick: use variable that you know will be small.
% Nondimensionalize first.

% Buckingham Pi theorem and nondimensionalization

Root-finding problems are hard or easy depending on how they are
posed.  Often, the initial problem formulation is not the most
convenient.  For example, consider the problem of finding the
positive root of
\[
  f(x) = (x+1)(x-1)^8-10^{-8}.
\]
This function is terrifyingly uninformative for values close to $1$.
Newton's iteration is based on the assumption that a local, linear
approximation provides a good estimate of the behavior of a function.
In this problem, a linear approximation is terrible.  Fortunately,
the function
\[
  g(x) = (x+1)^{1/8} (x-1)-10^{-1}
\]
has the same root, which is very nicely behaved.

There are a few standard tricks to make root-finding problems easier:
\begin{itemize}
\item 
  Scale the function.  If $f(x)$ has a zero at $x_*$, so does $f(x)
  g(x)$; and sometimes we can analytically choose a scaling function
  to make the root finding problem easier.
\item
  Otherwise transform the function.  For example, in computational
  statistics, one frequently would like to maximize a likelihood function
  \[
    L(\theta) = \prod_{j=1}^n f(x_j; \theta)
  \]
  where $f(x; \theta)$ is a probability density that depends on some
  parameter $\theta$.  One way to do this would be find zeros of
  $L'(\theta)$, but this often leads to scaling problems (potential
  underflow) and other numerical discomforts.  The standard trick is
  to instead maximize the log-likelihood function
  \[
    \ell(\theta) = \sum_{j=1}^n \log f(x_j; \theta),
  \]
  often using a root finder for $\ell'(\theta)$.  This tends to be a
  much more convenient form, both for analysis and for computation.
\item
  Change variables.  A good rule of thumb is to pick variables that
  are naturally {\em dimensionless}\footnote{%
Those of you who are interested in applied mathematics more generally
should look up the Buckingham Pi Theorem --- it's a tremendously
useful thing to know about.
}
  For difficult problems, these dimensionless variables are often very
  small or very large, and that fact can be used to simplify the
  process of coming up with good initial guesses for Newton iteration.
\end{itemize}

\section*{Starting points}

% Use crude bounds for bracketing intervals (ideally, want things to
% be monotone between).  The issue of having lots of solutions.

All root-finding software requires either an initial guess at the
solution or an initial interval that contains the solution.  This
sometimes calls for a little cleverness, but there are a few standard
tricks:
\begin{itemize}
\item
  If you know where the problem comes from, you may be able to get a
  good estimate (or bounds) by ``application reasoning.''  This is
  often the case in physical problems, for example: you can guess the
  order of magnitude of an answer because it corresponds to some
  physical quantity that you know about.
\item
  Crude estimates are often fine for getting upper and lower bounds.
  For example, we know that for all $x > 0$,
  \[
    \log(x) \leq x-1
  \]
  and for all $x \geq 1$, $\log(x) > 0$.  So if I wanted to $x +
  \log(x) = c$ for $c > 1$, I know that $c$ should fall between $x$
  and $2x-1$, and that gives me an initial interval.  Alternatively,
  if I know that $g(z) = 0$ has a solution close to $0$, I might try
  Taylor expanding $g$ about zero -- including higher order terms if
  needed -- in order to get an initial guess for $z$.
\item
  Sometimes, it's easier to find local minima and maxima than to find
  zeros.  Between any pair of local minima and maxima, functions are
  either monotonically increasing or monotonically decreasing, so
  there is either exactly one root in between (in which case there is a sign
  change between the local min and max) or there are zero roots
  between (in which case there is no sign change).  This can be a
  terrific way to start bisection.
\end{itemize}

\section*{Problems to ponder}

\begin{enumerate}
\item
  Analyze the convergence of the fixed point iteration
  \[
    x_{k+1} = c - \log(x_k).
  \]
  What is the equation for the fixed point?  Under what conditions
  will the iteration converge with a good initial guess, and at what
  rate will the convergence occur?
\item
  Repeat the previous exercise for the iteration 
  $x_{k+1} = 10-\exp(x_k)$.

\item
  Analyze the convergence of Newton's iteration on the equation 
  $x^2 = 0$, where $x_0 = 0.1$.  How many iterations will it take to get
  to a number less than $10^{-16}$?

\item
  Analyze the convergence of the fixed point iteration
  $x_{k+1} = x_k-\sin(x_k)$ for $x_k$ near zero.  Starting from 
  $x = 0.1$, how many iterations will it take to get to a number 
  less than $10^{-16}$?

\item
  Consider the cubic equation
  \[
    x^3 - 2 x + c = 0.
  \]
  Describe a general purpose strategy for finding {\em all} the real roots
  of this equation for a given $c$.

\item
  Suppose we have some small number of samples $X_1, \ldots, X_m$
  drawn from a Cauchy distribution with parameter $\theta$ (for which
  the pdf is)
  \[
    f(x,\theta) = \frac{1}{\pi} \frac{1}{1+(x-\theta)^2}.
  \]
  The {\em maximum likelihood estimate} for $\theta$ is the function
  that maximizes
  \[
    L(\theta) = \prod_{j=1}^m f(X_j,\theta).
  \]
  Usually, one instead maximizes $l(\theta) = \log L(\theta)$ --- why would this
  make sense numerically?  Derive a MATLAB function to find the
  maximum likelihood estimate for $\theta$ by finding an appropriate
  solution to the equation $l'(\theta) = 0$.

\item 
  The Darcy friction coefficient $f$ for turbulent flow in a pipe is 
  defined in terms of the Colebrook-White equation for large
  Reynolds number $\mathrm{Re}$ (greater than 4000 or so):
  \[
    \frac{1}{\sqrt{f}} = -2 \log_{10}\left(
      \frac{\epsilon/D_h}{3.7} + \frac{2.51}{\mathrm{Re}{\sqrt{f}}}
      \right)
  \]
  Here $\epsilon$ is the height of the surface roughness and $D_h$ is
  the diameter of the pipe.  For a 10 cm pipe with 0.1 mm surface
  roughness, find $f$ for Reynolds numbers of $10^4$, $10^5$, and
  $10^6$.  Ideally, you should use a Newton iteration with a good initial guess.

\item
  A cable with density of 0.52 lb/ft is suspended between towers of
  equal height that are 500 ft apart.  If the wire sags by 50 ft in
  between, find the maximum tension $T$ in the wire.  The relevant
  equations are
  \begin{align*}
    c + 50 &= c \cosh\left( \frac{500}{2c} \right) \\
    T &= 0.52(c + 50)
  \end{align*}
  Ideally, you should use a Newton iteration with a good initial guess.
\end{enumerate}

\end{document}
