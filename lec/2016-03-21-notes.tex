\documentclass[12pt, leqno]{article}
\usepackage{amsfonts}
\usepackage{amsmath}
\usepackage{amssymb}
\usepackage{fancyhdr}
\usepackage{hyperref}
\usepackage{tikz}
\usepackage{pgfplots}
\usepackage{listings}

\newcommand{\bbR}{\mathbb{R}}
\newcommand{\bbC}{\mathbb{C}}
\newcommand{\calV}{\mathcal{V}}
\newcommand{\calW}{\mathcal{W}}
\newcommand{\ddiag}{\operatorname{diag}}
\newcommand{\fl}{\operatorname{fl}}
\newcommand{\macheps}{\epsilon_{\mathrm{mach}}}
\newcommand{\matlab}{\textsc{Matlab}}

\newcommand{\hdr}[2]{
  \pagestyle{fancy}
  \lhead{Bindel, Spring 2016}
  \rhead{Numerical Analysis (CS 4220)}
  \fancyfoot{}
  \begin{center}
    {\large{\bf Notes for #1}}
  \end{center}
  \lstset{language=matlab,columns=flexible}  
}


\newcommand{\calK}{\mathcal{K}}
\newcommand{\calP}{\mathcal{P}}
\newcommand{\calR}{\mathcal{R}}

\begin{document}
\hdr{2016-03-21}

\section*{Direct to Iterative}

For the first part of the semester, we discussed {\em direct} methods for
solving linear systems and least squares problems.  These methods
typically involve a factorization, such as LU or QR, that reduces the
problem to a triangular solve using forward or backward substitution.
These methods run to completion in a fixed amount of time, and are
backed by reliable software in packages like LAPACK or UMFPACK.

There are a few things you need to know to be an informed {\em user}
(not developer) of direct methods:
\begin{itemize}
\item
  You need some facility with matrix algebra, so that you know how to
  manipulate matrix factorizations and ``push parens'' in order to
  compute efficiently.
\item
  You need to understand the complexity of different factorizations,
  and a little about how to take advantage of common matrix structures
  (e.g. low-rank structure, symmetry, orthogonality, or sparsity) in
  order to effectively choose between factorizations and algorithms.
\item
  You need to understand a little about conditioning and the
  relationship between forward and backward error.  This is important
  not only for understanding rounding errors, but also for
  understanding how other errors (such as measurement errors) can
  affect a result.
\end{itemize}
It's also immensely helpful to understand a bit about how the methods
work in practice.  On the other hand, you are unlikely to have to
build your own dense Gaussian elimination code with blocking for
efficiency; you'll probably use a library routine instead.  It's more
important that you understand the ideas behind the factorizations, and
how to apply those ideas to use the factorizations effectively in
applications.

Compared to direct methods, iterative methods provide more room for
clever, application-specific twists and turns.  An iterative method
for solving the linear system $Ax = b$ produces a series of guesses
\[
  \hat{x}^1, \hat{x}^2, \ldots \rightarrow x.
\]
The goal of the iteration is not always to get the exact answer as
fast as possible; it is to get a good enough answer, fast enough to be
useful.  The rate at which the iteration converges to the solution
depends not only on the nature of the iterative method, but also on
the structure in the problem.  The picture is complicated by the fact
that different iterations cost different amounts per step, so a
``slowly convergent'' iteration may in practice get an adequate
solution more quickly than a ``rapidly convergent'' iteration, just
because each step in the slowly convergent iteration is so cheap.

As with direct methods, though, sophisticated iterative methods are
constructed from simpler building blocks.  In this lecture, we set up
one such building block: stationary iterations.

\section*{Stationary Iterations}

A stationary iteration for the equation $Ax = b$ is typically
associated with a {\em splitting} $A = M-N$, where $M$ is a matrix
that is easy to solve (i.e. a triangular or diagonal matrix) and $N$
is everything else.  In terms of the splitting, we can rewrite
$Ax = b$ as
\[
  Mx = Nx + b,
\]
which is the fixed point equation for the iteration
\[
  Mx^{k+1} = Nx^{k} + b.
\]
If we subtract the fixed point equation from the iteration equation,
we have the error iteration
\[
  M e^{k+1} = N e^k
\]
or
\[
  e^{k+1} = R e^k, \quad R = M^{-1} N.
\]
We've already seen one example of such an iteration (iterative
refinement with an approximate factorization); in other cases,
we might choose $M$ to be the diagonal part of $A$ (Jacobi iteration)
or the upper or lower triangle of $A$ (Gauss-Seidel iteration).
We will see in the next lecture that there is an alternate
``matrix-free'' picture of these iterations that makes sense in the
context of some specific examples, but for analysis it is often best
to think about the splitting picture.

\section*{Convergence: Norms and Eigenvalues}

We consider two standard approaches to analyzing the convergence of a
stationary iteration, both of which revolve around the error iteration
matrix $R = M^{-1} N$.  These approaches involve taking a norm
inequality or using an eigenvalue decomposition.  The first approach
is often easier to reason about in practice, but the second is
arguably more informative.

For the norm inequality, note that if $\|R\| < 1$ for some operator
norm, then the error satisfies
\[
  \|e^{k+1}\| \leq \|R\| \|e^k\| \leq \|R\|^k \|e^0\|.
\]
Because $\|R\|^k$ converges to zero, the iteration eventually
converges.  As an example, consider the case where $A$ is
{\em strictly row diagonally dominant} (i.e.~the sum of the
magnitudes of the off-diagonal elements in each row are less
than the magnitude of the diagonal element), and let $M$ be the diagonal
part of $A$ (Jacobi iteration).  In that case,
$\|R\|_\infty = \|M^{-1} N\|_\infty < 1$.  Therefore, the infinity
norm of the error is monontonically decreasing\footnote{%
  In finite-dimensional spaces, there is a property of ``equivalence
  of norms'' that says that convergence in one norm implies
  convergence in any other norm; however, this does {\em not} mean
  that monotone convergence in one norm implies monotone convergence
  in any other norm.}

Bounding by one the infinity norm (or two norm, or one norm) of the
iteration matrix $R$ is {\em sufficient} to guarantee convergence,
but not {\em necessary}.  In order to completely characterize when
stationary iterations converge, we need to turn to an eigenvalue
decomposition.  Suppose $R$ is diagonalizable, and write the
eigendecomposition as
\[
  R = V \Lambda V^{-1}.
\]
Now, note that $R^k = V \Lambda^k V^{-1}$, and therefore
\[
  \|e^k\| = \|R^k e^0\| = \|V \Lambda^k V^{-1} e^0\| \leq \kappa(V)
  \rho(R)^k \|e^0\|,
\]
where $\rho(R)$ is the {\em spectral radius} of $R$, i.e.
\[
  \rho(R) = \max_{\lambda \mbox{ an eig}} |\lambda|,
\]
and $\kappa(V) = \|V\| \|V^{-1}\|$.  For a diagonalizable matrix,
convergence of the iteration happens if and only if the spectral
radius of $R$ is less than one.  {\em But} that statement ignores
the condition number of the eigenvector matrix!  For highly
``non-normal'' matrices in which the condition number is large,
the iteration may appear to make virtually no progress for many steps
before eventually it begins to converge at the rate predicted by
the spectral radius.  This is consistent with the bounds that we
can prove, but often surprises people who have not seen it before.

\section*{Splittings and Sweeps}

Splitting is the right linear algebraic framework for discussing
convergence of stationary methods, but it is not the way they are
usually programmed.  The connection between a matrix splitting and a
``sweep'' of a stationary iteration like Gauss-Seidel or Jacobi
iteration is not always immediately obvious, and so it is probably
worth spending a moment or two explaining in more detail.

For the sake of concreteness, let's consider a standard model problem:
a discretization of a Poisson equation on a line.  That is, we
approximate
\[
  -\frac{d^2 u}{dx^2} = f, \quad u(0) = u(1) = 0
\]
using the second-order finite difference approximation
\[
  \frac{d^2 u}{dx^2} \approx \frac{u(x-h)-2u(x)+u(x+h)}{h^2}
\]
where $h$ is a small step size.  We discretize the problem by meshing
$[0,1]$ with evenly spaced points $x_j = jh$ for $j = 0$
to $N+1$ where $h = 1/(N+1)$, then apply this approximation at each
point.  This procedure yields the equations
\[
  -u_{j-1}+2u_j-u_{j+1} = h^2 f_j, \quad j = 1, \ldots, N
\]
or, in matrix notation
\[
  Tu = h^2 f
\]
where $u$ and $f$ are vectors in $\bbR^N$ representing the sampled
(approximate) solution and the sampled forcing function.  The matrix
$T$ is a frequently-recurring model matrix, the tridiagonal
\[
T =
\begin{bmatrix}
   2 & -1 \\
  -1 &  2 & -1 \\
     & -1 &  2 & -1 \\
     &    & \ddots & \ddots & \ddots \\
     &    &        & -1 & 2 & -1 \\
     &    &        &    & -1 & 2
\end{bmatrix}.
\]

Suppose we forgot about how cheap Gaussian elimination is for
tridiagonal matrices.  How might we solve this system of equations?
A natural thought is that we could make an initial guess at the
solution, then refine the solution by ``sweeping'' over each node $j$
and adjusting the value at that node ($u_j$) to be consistent with
the values at neighboring nodes.  In one sweep, we might compute
a new set of values $u^{\mathrm{new}}$ from the old values $u^{\mathrm{old}}$:
\begin{lstlisting}
  for j = 1:N
    unew(j) = (h^2*f(j) + uold(j-1) + uold(j+1))/2;
  end
\end{lstlisting}
or we might update the values for each node in turn, using the most
recent estimate for each update, i.e.
\begin{lstlisting}
  for j = 1:N
    u(j) = (h^2*f(j) + u(j-1) + u(j+2))/2;
  end
\end{lstlisting}
These are, respectively, a step of {\em Jacobi} iteration and a step
of {\em Gauss-Seidel} iteration, which are two standard stationary
methods.

How should we relate the ``sweep'' picture to a matrix splitting?
The update equation from step $k$ to step $k+1$ in Jacobi is
\[
  -u_{j-1}^{(k)}+2u_j^{(k+1)}-u_{j+1}^{(k)} = h^2 f_j, 
\]
while the Gauss-Seidel update is
\[
  -u_{j-1}^{(k+1)}+2u_j^{(k+1)}-u_{j+1}^{(k)} = h^2 f_j.
\]
In terms of splittings, this means that Jacobi corresponds to
taking $M$ to be the diagonal part of the matrix,
\[
M =
\begin{bmatrix}
   2 &    \\
     &  2 &    \\
     &    &  2 &    \\
     &    &        & \ddots &  \\
     &    &        &    & 2 &  \\
     &    &        &    &   & 2
\end{bmatrix}, ~~
N =
\begin{bmatrix}
   0 &  1 \\
   1 &  0 &  1 \\
     &  1 &  0 & 1 \\
     &    & \ddots & \ddots & \ddots \\
     &    &        & 1 & 0 & 1 \\
     &    &        &   & 1 & 0
\end{bmatrix},
\]
while Gauss-Seidel corresponds to taking $M$ to be the lower triangle
of the matrix,
\[
M =
\begin{bmatrix}
   2 &  \\
  -1 &  2 &  \\
     & -1 &  2 &  \\
     &    & \ddots & \ddots &  \\
     &    &        & -1 & 2 &  \\
     &    &        &    & -1 & 2
\end{bmatrix}, ~~
N =
\begin{bmatrix}
   0 &  1 \\
     &  0 &  1 \\
     &    &  0 & 1 \\
     &    &    & \ddots & \ddots \\
     &    &        &  & 0 & 1 \\
     &    &        &    &   & 0
\end{bmatrix}.
\]

The point of this exercise is that {\em programming} stationary
iterative methods and {\em analyzing} the same methods may lead
naturally to different ways of thinking about the iterations.
It's worthwhile practicing mapping back and forth between these
two modes of thought.

\section*{Linear Solves and Quadratic Minimization}

We have already briefly described an argument that Jacobi iteration
converges for strictly row diagonally dominant matrices.  We now
discuss an argument that Gauss-Seidel converges (or at least part of
such an argument).  In the process, we will see a useful way of
reformulating the solution of symmetric positive definite linear
systems that will prepare us for our upcoming discussion of conjugate
gradient methods.

Let $A$ be a symmetric positive definite matrix, and consider
the ``energy'' function
\[
  \phi(x) = \frac{1}{2} x^T A x - x^T b.
\]
The stationary point for this function is the point at which
the derivative in any direction is zero.  That is, for any
direction vector $u$,
\begin{align*}
  0
  &=\left. \frac{d}{d\epsilon} \right|_{\epsilon = 0} \phi(x+\epsilon u) \\
  &=\frac{1}{2} u^T A x + \frac{1}{2} x^T A u -  u^T b \\
  &=u^T (Ax-b)
\end{align*}
Except in pathological instances, a directional derivative can be
written as the dot product of a direction vector and a gradient;
in this case, we have
\[
  \nabla \phi = Ax-b.
\]
Hence, minimizing $\phi$ is equivalent to solving $Ax = b$~\footnote{%
If you are unconvinced that this is a minimum, work through the
algebra to show that $\phi(A^{-1} b + w) = \frac{1}{2} w^T A w$ for any $w$.}.

Now that we have a characterization of the solution of $Ax = b$
in terms of an optimization problem, what can we do with it?
One simple approach is to think of a sweep through all the unknowns,
adjusting each variable in term to minimize the energy; that is,
we compute a correction $\Delta x_j$ to node $j$ such that
\[
  \Delta x_j = \operatorname{argmin}_z \phi(x+z e_j)
\]
Note that
\[
  \frac{d}{dz} \phi(x+z e_j) = e_j^T (A(x+ze_j)-b),
\]
and the update $x_j := x_j + \Delta x_j$ is equivalent to choosing
a new $x_j$ to set this derivative equal to zero.  But this is
exactly what the Gauss-Seidel update does!  Hence, we can see
Gauss-Seidel in two different ways: as a stationary method for solving
a linear system, or as an optimization method that constantly makes
progress toward a solution that minimizes the energy~\footnote{%
Later in the class, we'll see this as coordinate-descent with
exact line search.}.
The latter perspective can be turned (with a little work) into
a convergence proof for Gauss-Seidel on positive-definite linear systems.

\section*{Extrapolation: A Hint of Things to Come}

Stationary iterations are simple.  Methods like Jacobi or Gauss-Seidel
are easy to program, and it's (relatively) easy to analyze their
convergence.  But these methods are also often slow.  We'll talk next
time about more powerful {\em Krylov subspace} methods that use
stationary iterations as a building block.

There are many ways to motivate Krylov subspace methods.  We'll 
pick one motivating idea that extends beyond
the land of linear solvers and into other applications as well.
The key to this idea is the observation that the error in our
iteration follows a simple pattern:
\[
  x^{(k)}-x = e^{(k)} = R^k e^{(0)}, \quad R = M^{-1} N.
\]
For large $k$, the behavior of the error is dominated by the largest
eigenvalue and the associated eigenvector\footnote{%
If you don't understand this now, you will when we talk about the
power method in a week or so!}, i.e.
\[
  e^{(k+1)} \approx \lambda_1 e^{(k)}.
\]
Note that this means
\[
  x^{(k)}-x^{(k+1)} = e^{(k)}-e^{(k+1)} \approx (1-\lambda_1) e^{(k)}.
\]
If we have an estimate for $\lambda_1$, we can write
\[
  x = x^{(k)} - e^{(k)} \approx
  x^{(k)}-\frac{x^{(k)}-x^{(k+1)}}{1-\lambda_1}.
\]
That is, we might hope to get a better estimate of $x$ than is
provided by $x^{(k)}$ or $x^{(k+1)}$ individually by taking an
appropriate linear combination of $x^{(k)}$ and $x^{(k+1)}$.  This
idea generalizes: if we have a sequence of approximations
$x^{(0)}, \ldots, x^{(k)}$, why not ask for the ``best'' approximation
that can be written as a linear combination of the $x^{(j)}$?
This is the notion underlying methods such as the conjugate iteration
iteration, which we will discuss shortly.

\section*{The Big Picture}

We now start with our discussion of Krylov subspace
methods in general, and the famous method of conjugate gradients (CG)
in particular.  Though this is the iterative method of choice for most
positive definite systems, it may be as famously confusing as it is
famous\footnote{%
  See, e.g., ``Introduction to the Conjugate Gradient
  Method Without the Agonizing Pain'' by Jonathan Shewchuk.}.
In order to avoid getting lost in the weeds, it seems worthwhile to
start with a roadmap:
\begin{itemize}
\item
  We begin with the observation that if vectors $x^{(0)}, \ldots,
  x^{(k)}$ are increasingly good approximations to $x$, then some
  linear combination of these vectors may produce an even better
  approximation.  If the original sequence is produced by a stationary
  iteration, these vectors span a {\em Krylov subspace}.
\item
  One can generally show that a big enough Krylov subspace will
  contain a good approximation to $x$.  Alas, this does not tell us
  how to find which vector in the space is best (or even good)!
  Attempting to minimize the norm of the error is usually impossible,
  but it is possible to minimize the residual (leading to GMRES or
  MINRES), an energy function (CG), or some other error-related
  quantity.
\item
  The basic framework of a Krylov subspace plus a method of choosing
  approximations from the space allows us to describe some theoretical
  properties of several iterations without telling us why (or if) we
  can implement them efficiently and stably.  A key practical point is
  the computation of well-conditioned bases for the Krylov subspaces,
  e.g., using the {\em Lanczos} algorithm (symmetric case) or the
  {\em Arnoldi} algorithm (nonsymmetric case).
\end{itemize}

\section*{From Stationary Methods to Krylov Subspaces}

Earlier in these notes, we tried to motivate the idea that
we can improve the convergence of a stationary method by replacing
the sequence of guesses
\[
  x^0, x^1, \ldots \rightarrow x
\]
with {\em linear combinations}
\[
  \tilde{x}^k = \sum_{j=1}^k \alpha_{kj} x^j.
\]
We could always choose $\alpha_{kj}$ to be one for $k = j$ and zero
otherwise, in which case we have the original stationary method;
but by choosing the coefficients more carefully, we might do better.

We've so far written stationary methods as
\[
  M x^{j+1} = N x^j + b.
\]
This is equivalent to
\[
  x^{j+1} = x^j + M^{-1} r^j, \quad r^j \equiv b-Ax^j,
\]
or
\[
  x^{j+1} = R x^j + M^{-1} b
\]
where $R = I-M^{-1} A$ is the iteration matrix we've seen
in our previous analysis.  If $x^0 = M^{-1} b$, this gives
\[
  x^j = \sum_{i=1}^j R^i M^{-1} b.
\]
If we look at this expression closely, we might notice that
the space spanned by the first $k$ iterates of the stationary
method is all vectors of the form
\[
  \sum_{i=0}^j c_i R^i M^{-1} b.
\]
If we look a little harder, we might observe that this is equivalent
to the space of all vectors of the form
\[
  \sum_{i=0}^j c_i (M^{-1} A)^i M^{-1} b = p(M^{-1} A) M^{-1} b
\]
where $p(z) = \sum_{i=1}^j c_i z^i$ is a polynomial of degree at most
$j$.

In general, the $d$-dimensional Krylov subspace generated by
a matrix $A$ and vector $B$ is
\begin{align*}
  \calK_d(A,b)
  &= \operatorname{span}\{ b, Ab, A^2 b, \ldots, A^{d-1} b \} \\
  &= \{ p(A) b : p \in \calP_{d-1} \}.
\end{align*}
As we have just observed, the iterates of a stationary method
form a basis for nested Krylov subspaces generated by $M^{-1}A$
and $M^{-1}b$.  If the stationary method converges, we know
the Krylov subspaces will eventually contain pretty good
approximations to $A^{-1} b$.  Let's now spell this out a little
more carefully.

\section*{The Power of Polynomials}

We showed a moment ago that the first $m$ iterates of a stationary
method form a basis for the space
\[
  \calK_{m+1}(R, M^{-1}b) = \calK_{m+1}(M^{-1} A, M^{-1} b)
\]
What can we say about the quality of this space for approximation?  As
it turns out, this turns into a question about polynomial
approximations.  We will not spell out all the details (nor will this
appear on any exams or homework problems for this class), but it's
worth spending a few moments giving an idea of what happens.

We have seen that the iterates of the stationary method are
\[
  x^{(k)} = x + e^{(k)} = x + R^k e^{(0)}
\]
We would like to take a linear combination
\[
  \tilde x^{(m)} = \sum_{k=0}^m \gamma_{mk} x^{(k)} = p_m(1) x + p_m(R) e^{(0)}
\]
where $p_m(z) = \sum_{k=0}^d \gamma_{mk} z^k$.  Moreover,
if $R$ is diagonalizable with $R = V \Lambda V^{-1}$, then
\[
  p_m(R) = V p(\Lambda) V^{-1}.
\]
For any $p_m$ with $p_m(1) = 1$, we have
\begin{align*}
\|\tilde{e}^{(m)}\|
&= \|\tilde{x}^{(m)}-x\| \\
&= \|p_m(R) e^{(0)}\| = \|V p(\Lambda) V^{-1} e^{(0)}\| \\
&\leq \kappa(V) \max_{\lambda_j} |p(\lambda_j)| \|e^{(0)}\|.
\end{align*}
Hence, we would really like to choose the polynomial that is one
at $1$ and as small as possible on each of the eigenvalues.

If all eigenvalues $\lambda_j$ of $R$ are real, then we have
\[
  \max_{\lambda_j} |p(\lambda_j)| \leq \max_{|z|<\rho(R)} |p(z)|,
\]
and a reasonable way to choose polynomials is to minimize $|p_m(z)|$
on $[-\rho(R),\rho(R)]$ subject to the constraint $p_m(1) = 1$.  The
solution to this problem is the {\em scaled Chebyshev polynomials},
with which we can show that the optimal $p_m$ satisfies
\begin{align*}
  p_m(z)
  & \leq \frac{2}{1+m\sqrt{2/(1-\rho(R))}} \\
  & = 2(1-m\sqrt{2(1-\rho(R))}) + O(m^2(1-\rho(R)).
\end{align*}
While the number of steps for the basic stationary iteration to
reduce the error by a fixed amount scales roughly like
$(1-\rho(R))^{-1}$, the number of steps to reduce the bound on the
optimal error scales like $(1-\rho(R))^{-1/2}$.

While the Chebyshev bounds are correct, and involve a beautiful bit of
approximation theory, they are limited.  For one thing, they fall
apart on non-normal matrices with complex spectra.  Even in the SPD
case, these bounds are often highly pessimistic in practice.  When the
eigenvalues of $R$ come in clusters, a relatively low degree
polynomial can do a good job of approximating $\lambda^{-1}$ at each
of the clusters, and hence a relatively low-dimensional Krylov
subspace may provide excellent solutions.

All of this is to say that the detailed convergence theory for Krylov
subspace methods can be quite complicated, but understanding a little
about the eigenvalues of the problem can provide at least a modicum of
insight.

For theoretical work, we are fine writing Krylov subspaces as
polynomials in $R$ applied to $M^{-1} b$.  In practical computations,
though, we need a basis.  Because the iterates of the stationary
method are converge to $x$, they tend to form a very ill-conditioned
basis.  We would like to keep the same Krylov subspace, but have a
different basis -- say, for instance, an orthonormal basis.  We turn
to this task next.

\section*{The Lanczos Idea}

What is good about the ``power basis'' for a Krylov subspace?  That is,
why might we like to write
\[
  \calK_m(A,b) = \operatorname{span}\{ b, Ab, A^2b, \ldots, A^{m-1}b \}
\]
rather than choosing a different basis?  Though it's terrible for
numerical stability, there are two features of the basis that are
attractive:
\begin{itemize}
\item
  The power bases span nested Krylov subspaces.  Given the vectors
  $b, \ldots, A^{m-1} b$ spanning $\calK_m(A,b)$, we only need one
  more vector ($A^d b$) to span $\calK_{m+1}(A,b)$.
\item
  Each successive vector can be computed from the previous vector
  with a single multiplication by $A$.  There is no other overhead.
\end{itemize}
While we dislike the power basis from the perspective of stability,
we would like to keep these attractive features for any alternative
basis.

We've already described one approach to converting the vectors in
a power basis into a more convenient set of vectors.  Define the matrix
\[
  X^{(m)} = \begin{bmatrix} b & Ab & A^2 b & \ldots & A^{m-1} b \end{bmatrix}
\]
and consider the economy QR decomposition
\[
X^{(m)} = Q^{(m)} R^{(m)}, \quad
Q^{(m)} = \begin{bmatrix} q_1 & q_2 & \ldots & q_{m} \end{bmatrix}.
\]
The columns of $Q^{(m)}$ are orthonormal and for any $k \leq m$, the
first $k$ columns of $Q^{(m)}$ span the same space as the first $k$
columns of $X^{(m)}$.  But forming $X^{(m)}$ and running QR is
unattractive for two reasons.  First, a dense QR decomposition may
involve a nontrivial amount of extra work, particularly as $m$ gets
large; and second, simply forming the rounded version of $X^{(m)}$ is
enough to get us into numerical trouble, even if we were able to run
QR with no additional rounding errors.  We need a better approach.

One helpful observation is that
\[
  \calR(AQ^{(k)}) = \calR(AX^{(k)}) \subseteq
  \calR(X^{(k+1)}) = \calR(Q^{(k+1)}).
  \]
That is, $Aq_k$ can always be written as a linear combination
of $q_1, \ldots, q_{k+1}$.  In matrix terms, this means we can write
\[
  AQ^{(k)} = Q^{(k+1)} \bar{H}^{(k)}
\]  
where
\[
\bar{H}^{(m)} =
\begin{bmatrix}
  h_{11} & h_{12} & h_{13} & \ldots & h_{1,k-1} & h_{1k} \\
  h_{21} & h_{22} & h_{23} &        & h_{2,k-1} & h_{2k} \\
  0     & h_{32} & h_{33} &        & h_{3,k-1} & h_{3k} \\
        & 0      & h_{43} &       & h_{4,k-1} & h_{4k} \\
        &        & \ddots & \ddots & \vdots & \vdots \\
        &        &        & 0      &  h_{k,k-1} & h_{kk} \\
        &        &        &        &  0        & h_{k+1,k}
\end{bmatrix}.
\]
A matrix with this structure (all elements below the first subdiagonal
equal to zero) is called {\em upper Hessenberg}.  Alternately,
we write
\[
  AQ^{(k)} = Q^{(k)} H^{(k)} + q_{k+1} h_{k+1,k}
\]
where $H^{(k)}$ is the square matrix consisting of all but the last
row of $\bar{H}^{(k)}$.  This formula is sometimes called an
{\em Arnoldi decomposition}; it turns out to be crucial in the
development of GMRES, one of the most popular iterative solvers for
nonsymmetric linear systems.  For the moment, though, we want to
focus on the symmetric case, and so we will move on.

When $A$ is symmetric, we have that
\[
  (Q^{(k)})^T A Q^{(k)} = H^{(k)}
\]
is also symmetric, as well as being upper Hessenberg; in this case,
the matrix $H^{(k)}$ is actually {\em tridiagonal}, and we write
$H^{(k)}$ as $T^{(k)}$ to emphasize this fact.  Conventionally,
$T^{(k)}$ is written as
\[
T^{(k)} =
\begin{bmatrix}
  \alpha_1 & \beta_1 \\
  \beta_1 & \alpha_2 & \beta_2 \\
          & \beta_2 & \alpha_3 & \ddots \\
          &         & \ddots & \ddots & \beta_{k-1} \\
          &         &        & \beta_{k-1} & \alpha_k
\end{bmatrix}.
\]
Converting from matrix notation back into vector notation, we have
\[
  A q_{k} = \beta_{k-1} q_{k-1} + \alpha_k q_k + \beta_k q_{k+1},
\]
which we can rearrange as
\[
  \beta_k q_{k+1} = A q_k - \alpha_k q_k - \beta_{k-1} q_{k-1}. 
\]
where
\begin{align*}
  \alpha_k &= q_k^T A q_k \\
  \beta_{k-1} &= q_k^T A q_{k-1} = q_{k-1}^T A q_k.
\end{align*}
Putting everything together, we have the {\em Lanczos iteration},
in which we obtain each successive vector $q_{k+1}$ by forming $Aq_k$,
orthogonalizing against the $q_k$ and $q_{k-1}$ by Gram-Schmidt,
and normalizing.  We saw this iteration earlier in the semester
when we talked about eigenvalue problems and compared different
methods of matrix tridiagonalization.

Presented as a {\em fait accompli}, the Lanczos iteration looks like
magic.  It seems even more like magic when one realizes that despite
an instability in the iteration (because of the use of Gram-Schmidt
for orthonormalization at each step), the iteration still produces
useful information in floating point.  The Lanczos iteration is the
basis for one of the most popular iterative methods for solving
eigenvalue problems, and in that setting it is important to
acknowledge and deal with the instability in the method.  For the
moment, though, we are still interested in solving linear systems,
and the method of Conjugate Gradients (also built on Lanczos)
turns out to still work great.

\section{Addendum: Three-Term Recurrences}

The Lanczos iteration allows us to generate a sequence of orthonormal
vectors using a three-term recurrence.  As it turns out, the same
approach leads to three-term recurrences that generate families of
orthogonal polynomials, including the Chebyshev polynomials mentioned
in passing earlier and the Legendre polynomials that play
a significant role in the development of Gaussian quadrature.  I
consider the details beyond of these connections to be beyond the
scope of the current class.  But the connections are too beautiful and
numerous to not mention that they exist.  I would hate for you to walk
away from this class with the impression that the mathematical
development of the Lanczos iteration is only some quirky trick in
numerical linear algebra that gets you part of the way to CG.

\section*{From Lanczos to CG}

We developed the Lanczos iteration, which for a symmetric matrix $A$
implicitly generates the decomposition
\[
  A Q^{(k)} = Q^{(k)} T^{(k)} + \beta_k q_{k+1}
\]
where $T^{(k)}$ is a tridiagonal matrix with $\alpha_1, \ldots,
\alpha_k$ on the diagonal and $\beta_1, \ldots, \beta_{k-1}$ on
the subdiagonal and superdiagonal.  The columns of $Q^{(k)}$ form
an orthonormal basis for the Krylov subspace $\calK_{k}(A,b)$,
and are a numerically superior alternative to the power basis.
We now turn to using this decomposition to solve linear systems.

The {\em conjugate gradient} algorithm can be characterized as
a method that chooses an approximation
$\tilde{x}^{(k)} \in \calK_k(A,b)$ by minimizing the energy
function
\[
  \phi(z) = \frac{1}{2} z^T A z - z^T b
\]
over the subspace.  Writing $\tilde{x}^{(k)} = Q^{(k)} u$,
and using the fact that
\begin{align*}
  (Q^{(k)})^T A Q^{(k)} & = T^{(k)} \\
  (Q^{(k)})^T b &= \|b\| e_1
\end{align*}
we have
\[
  \phi(Q^{(k)} u) = \frac{1}{2} u^T T^{(k)} u - \|b\| u^T e_1.
\]
The stationary equations in terms of $u$ are then
\[
  T^{(k)} u = \|b\| e_1.
\]

In principle, we could solve find the CG solution by forming and
solving this tridiagonal system at each step, then taking an
appropriate linear combination of the Lanczos basis vectors.
Unfortunately, this would require that we keep around the Lanczos
vectors, which eventually may take quite a bit of storage.
This is essentially what happens in methods like GMRES, but
for the method of conjugate gradients, we have not yet exhausted
our supply of cleverness.  It turns out that we can derive a short
recurrence relating the solutions at consecutive steps and their
residuals.  There are several different ways to this recurrence:
one can work from a factorization of the nested tridiagonal matrices
$T^{(k)}$, or work out the recurrence based on the optimization
interpretation of the problem (this leads to the name ``conjugate
gradients'').  For a detailed discussion, my preferred reference
is {\em Templates for the Solution of Linear Systems},
a short book available from SIAM which is also freely available
online.  The paper ``Introduction to the Conjugate Gradient
Method Without the Agonizing Pain'' provides a longer and possibly
gentler introduction.


\section*{Practical Matters}

CG is popular for several reasons:
\begin{itemize}
\item
  The only thing we need $A$ for is to form matrix-vector products.
  An implicit representation of $A$ may be sufficient for this
  (and is in some cases more convenient than an explicit version).
\item
  Because it involves a short recurrence, CG can be run for many steps
  without an excessive amount of storage or other overheads involving
  looking over many past vectors.  This is not true of all other CG
  methods.
\item
  Convergence is faster than with stationary methods.
\end{itemize}
This last point requires some clarification, and will take up most of
the rest of our discussion.

When we started talking about Krylov subspaces, we described searching
over the spaces $\calK_m(M^{-1} A, M^{-1} b)$, where $M$ is the matrix
appearing in the splitting for some stationary iteration.  When we
derived CG, though, the matrix $M$ disappeared.  In fact, it
disappeared only to keep the presentation as uncluttered as I could
manage.  In practice, CG and other Krylov subspace methods are
typically used with a {\em preconditioner} $M$, and one looks over
$\calK_m(M^{-1} A, M^{-1} b)$ for solutions.  The preconditioner
should have the following properties
\begin{itemize}
\item
  As in a splitting for a stationary method, it should be easy to
  apply $M^{-1}$.  There is a tradeoff here: the quality of the
  subspace and the efficiency with which $M^{-1}$ can be applied may
  be in conflict.  Note that solving systems involving $M$ is the only
  way in which $M$ is used (just as applying $A$ is the only way in
  which $A$ is used).
\item
  The preconditioner does not need to correspond to a convergent
  stationary iteration, but $M^{-1}$ should ``look like'' $A^{-1}$
  in that $M^{-1} A$ should have eigenvalues in clusters.  If all the
  eigenvalues are the same, preconditioned CG (or other preconditioned
  Krylov subspace methods) will converge in one step.
\item
  For CG, the preconditioner must be symmetric and positive definite.
\end{itemize}

Are preconditioners really necessary?  For problems coming from PDE
discretizations or computations on networks, there is sometimes an
intuitive way to see why the answer is ``yes'' if one wants fast
convergence.  Consider, for example, the model tridiagonal matrix $T$
in $\bbR^{N \times N}$, and imagine we want to solve $Tx = e_1$.  The
last component of $x$ is not that tiny, but notice that all the
vectors in a Krylov subspace $\calK_m(T, e_1)$ are zero for every
index after $m$.  Therefore, there is no way a method that explores
these Krylov subspaces will be close to converged for $m < N$
iterations.  Given that we know how to solve this tridiagonal system
in $O(N)$ time with Gaussian elimination, this is a bit disheartening!
The issue is that method simply doesn't move information through the
mesh fast enough to achieve rapid convergence.

One way of deriving preconditioners is to look at the splittings used
in classical stationary iterations.  Another approach is incomplete
factorization -- carry out Gaussian elimination or Cholesky, but
throw away nonzeros that are small or appear in inconvenient places.
However, the best preconditioners are often specific to particular
applications.  Some examples are:
\begin{itemize}
\item
  For discretizations of PDEs with variable coefficients, we might use
  a preconditioner based on a much more regular PDE (e.g. with
  constant coefficients and a regular geometry).  The preconditioner
  solve could potentially be applied by fast transform methods.
\item
  Multigrid preconditioners take advantage of the fact that one can
  often discretize a continuous problem using coarser or finer meshes,
  and the coarse meshes can be used to suppress parts of the error
  that are hard to reach by applying standard stationary methods to
  fine grids.
\item
  Domain decomposition preconditioners split the problem into
  subproblems.  If the subproblems are treated completely
  independently, domain decomposition looks like a block version of
  Jacobi iteration; but if the subproblems overlap, or if they are
  coupled together in some other way, one gets any of a variety of
  interesting methods (additive and multiplicative Schwarz, FETI,
  BDDC, etc).
\end{itemize}
Frameworks like PETSc and Trilinos provide both standard iterative
solvers like CG and a menu of different preconditioners.  Choosing the
right preconditioner is in general hard, and choosing the parameters
that determine the detailed behavior is hard.


\end{document}
