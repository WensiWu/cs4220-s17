\documentclass[12pt, leqno]{article} %% use to set typesize 
\usepackage{amsfonts}
\usepackage{amsmath}
\usepackage{amssymb}
\usepackage{fancyhdr}
\usepackage{hyperref}
\usepackage{tikz}
\usepackage{pgfplots}
\usepackage{listings}

\newcommand{\bbR}{\mathbb{R}}
\newcommand{\bbC}{\mathbb{C}}
\newcommand{\calV}{\mathcal{V}}
\newcommand{\calW}{\mathcal{W}}
\newcommand{\ddiag}{\operatorname{diag}}
\newcommand{\fl}{\operatorname{fl}}
\newcommand{\macheps}{\epsilon_{\mathrm{mach}}}
\newcommand{\matlab}{\textsc{Matlab}}

\newcommand{\hdr}[2]{
  \pagestyle{fancy}
  \lhead{Bindel, Spring 2016}
  \rhead{Numerical Analysis (CS 4220)}
  \fancyfoot{}
  \begin{center}
    {\large{\bf Notes for #1}}
  \end{center}
  \lstset{language=matlab,columns=flexible}  
}


\begin{document}
\hdr{2016-04-20}

\section*{Consider constraints}

So far, we have considered {\em unconstrained} optimization problems.
The {\em constrained} problem is
\[
  \mbox{minimize } \phi(x) \mbox{ s.t. } x \in \Omega
\]
where $\Omega \subset \bbR^n$.  We usually define $x$ in terms
of a collection of constraint equations and inequalities\footnote{
  I am never sure whether the standard form for inequality constraints
  is positive $c$ means infeasible, or negative $c$.  The book uses
  the former convention.  For these notes, I'll stick with the latter.
}:
\[
\Omega = \{ x \in \bbR^n :
  c_i(x) = 0, i \in \mathcal{E} \mbox{ and }
  c_i(x) \leq 0, i \in \mathcal{I} \}.
\]
We will suppose throughout our discussions that both
$\phi$ and all the functions $c$ are differentiable.

If $x_*$ is a solution to the constrained minimization problem, we say
constraint $i \in \mathcal{I}$ is {\em active} if $c_i(x) = 0$.
Often, the hard part of solving constrained optimization problems is
figuring out which constraints are active.  From this perspective, the
equality constrained problem sits somewhere in difficulty between the
unconstrained problem and the general constrained problem.

Our treatment of constrained optimization is necessarily brief;
but in the next two lectures, I hope to lay out some of the big
ideas.  Today we will focus on formulations; next time, algorithms.

\section*{Three recipes}

Most methods for constrained optimization involve a reduction to
an unconstrained problem (or subproblem).  There are three ways
such a reduction might work:
\begin{itemize}
\item We might {\em remove} variables by eliminating constraints.
\item We might keep the {\em same} number of variables and try to fold
  the constraints into the objective function.
\item We might {\em add} variables to enforce constraints via
  the method of Lagrange multipliers.
\end{itemize}
These approaches are not mutually exclusive, and indeed one often
alternates between perspectives in modern optimization algorithms.

\subsection*{Constraint elimination}

% Linear
% Nonlinear (non-negativity, sphere)
% Advantages (smaller system, retain convexity for linear case)
% Disadvantages (lose sparsity)
%
% Example: linear + quadratic; exponential and square transforms; etc

The idea of constraint elimination is straightforward.  Suppose we
want to solve an optimization problem with only equality constraints:
$c_i(x) = 0$ for $i \in \mathcal{E}$, where $|\mathcal{E}| < n$ and
the constraints are independent -- that is, the
$|\mathcal{E}| \times n$ Jacobiam matrix $\partial c / \partial x$
has full row rank.  Then we can think (locally) of $x$ satisfying the
constraints in terms of an implicitly defined function $x = g(y)$
for $y \in \bbR^{n-|\mathcal{E}|}$.  If this characterization can be
made global, then we can solve the unconstrained problem
\[
  \mbox{minimize } \phi(g(y))
\]
over all $y \in \bbR^{n-|\mathcal{E}|}$.

The difficulty with constraint elimination is that it requires that we
find a global parameterization of the solutions to the constraint
equations.  This is usually difficult.  An exception is when the
constraints are {\em linear}:
\[
  c(x) = A^T x - b
\]
In this case, the feasible set $\Omega = \{ x : A^T x - b = 0 \}$ can
be written as $x \in \{ x^p + z : z \in \mathcal{N}(A) \}$, where $x^p$
is a {\em particular solution} and $\mathcal{N}(A)$ is the null space
of $A$.  We can find both a particular solution and the null space by
doing a full QR decomposition on $A$:
\[
A = \begin{bmatrix} Q_1 & Q_2 \end{bmatrix}
    \begin{bmatrix} R_1 \\ 0 \end{bmatrix}.
\]
Then solutions to the constraint equations have the form
\[
  x = Q_1 R_1^{-T} b + Q_2 y
\]
where the first term is a particular solution and the second term
gives a vector in the null space.

For problems with linear equality constraints, constraint elimination
has some attractive properties.  If there are many constraints, the
problem after constraint elimination may be much smaller.  And if the
original problem was convex, then so is the reduced problem, and with
a better-conditioned Hessian matrix.  The main drawback is that we
may lose sparsity of the original problem.  Constraint elimination is
also attractive for solving equality-constrained subproblems in
optimization algorithms for problems with linear {\em inequality}
constraints, particularly if those constraints are simple
(e.g.~elementwise non-negativity of the solution vector).

For problems with more complicated equality constraints, constraint
elimination is hard.  Moreover, it may not be worthwhile; in some
cases, eliminating constraints results in problems that are smaller
than the original formulation, but are harder to solve.

The idea of constraint elimination is not limited to equality
constraints: one can also sometimes use an alternate parameterization
to convert simple inequality-constrained problems to unconstrained
problems.  For example, if we want to solve a non-negative
optimization problem (all $x_i \geq 0$), we might write $x_i = y_i^2$,
or possibly $x_i = \exp(y_i)$ (though in this case we would need to
let $y_i \rightarrow -\infty$ to exactly hit the constraint).  But
while they eliminate constraints, these re-parameterizations can also
destroy nice features of the original problem (e.g.~convexity).  So
while such transformations are a useful part of the computational
arsenal, they should be treated as one tool among many, and not always
as the best tool available.

\subsection*{Penalties and barriers}

% Basic idea: modify the function
% Penalty vs barrier and interior vs exterior point methods
% Exact penalty (not discussed in class)

Constraint elimination methods convert a constrained to an
unconstrained problem by changing the coordinate system in which the
problem is posed.  Penalty and barrier methods accomplish the same
reduction to the unconstrained case by changing the function.

As an example of a {\em penalty} method, consider the problem
\[
  \mbox{minimize } \phi(x) + \frac{1}{2\mu} \sum_{i\in \mathcal{E}}
  c_i(x)^2 + \frac{1}{2\mu} \sum_{i \in \mathcal{I}} \max(c_i(x),0)^2.
\]
When the constraints are violated ($c_i > 0$ for inequality
constraints and $c_i \neq 0$ for equality constraints), the extra
terms (penalty terms) beyond the original objective function are
positive; and as $\mu \rightarrow 0$, those penalty terms come to
dominate the behavior outside the feasible region.  Hence as we let
$\mu \rightarrow 0$, the solutions to the penalized problem approach
solutions to the original (true) problem.  At the same time, as $\mu
\rightarrow 0$ we have much wilder derivatives of $\phi$, and the
optimization problems become more and more problematic from the
perspective of conditioning and numerical stability.  Penalty methods
also have the potentially undesirable property that if any constraints
are active at the true solution, the solutions to the penalty problem
tend to converge from {\em outside} the feasible region.  This poses a
significant problem if, for example, the original objective function
$\phi$ is undefined outside the feasible region.

As an example of a {\em barrier} method, consider the purely inequality
constrained case, and approximate the original constrained problem
by the unconstrained problem
\[
  \mbox{minimize } \phi(x) - \mu \sum_{i \in \mathcal{I}} \log(-c_i(x)).
\]
As $c_i(x)$ approaches zero from below, the barrier term
$-\mu \log (-c_i(x))$ grows rapidly; but at any fixed $x$ in the
interior of the domain, the barrier goes to zero as $\mu$ goes to
zero.  Hence, as $\mu \rightarrow 0$ through positive values, the
solution to the barrier problem approaches the solution to the true
problem through a sequence of {\em feasible} points (i.e.~approximate
solutions that satisfy the constraints).  Though the feasibility of
the approximations is an advantage over penalty based formulations,
interior formulations share with penalty formulations the disadvantage
that the solutions for $\mu > 0$ lie at points with increasingly large
derivatives (and bad conditioning) if the true solution has active
constraints.

There are {\em exact penalty} formulations for which the solution to
the penalized problem is an exact solution for the original problem.
Suppose we have an inequality constrained problem in which the
feasible region is closed and bounded, each constraint $c_i$ has
continuous derivatives, and $\nabla c_i(x) \neq 0$ at any boundary
point $x$ where constraint $i$ is active.  Then the solution to the
problem
\[
  \mbox{minimize } \phi(x) + \frac{1}{\mu} \sum_i \max(c_i(x), 0)
\]
is {\em exactly} the solution to the original constrained optimization
problem for some $\mu > 0$.  In this case, we used a {\em nondifferentiable}
exact penalty, but there are also exact differentiable penalties.

\subsection*{Lagrange multipliers}

% Physical picture
% Mathematical picture
% KKT equations

Picture a function $\phi : \bbR^n \rightarrow \bbR$; if you'd like to
be concrete, let $n = 2$.  Absent a computer, we might optimize of
$\phi$ by the physical experiment of dropping a tiny ball onto the
surface and watching it roll downhill (in the steepest descent
direction) until it reaches the minimum.  If we wanted to solve a
constrained minimization problem, we could build a great wall between
the feasible and the infeasible region.  A ball rolling into the wall
would still roll freely in directions tangent to the wall (or away
from the wall) if those directions were downhill; at a constrained
miminizer, the force pulling the ball downhill would be perfectly
balanced against an opposing force pushing into the feasible region
in the direction of the normal to the wall.  If the feasible region
is $\{x : c(x) \leq 0\}$, the normal direction pointing inward at a
boundary point $x_*$ s.t.~$c(x_*) = 0$ is proportional to
$-\nabla c(x_*)$.  Hence, if $x_*$ is a constrained minimum,
we expect the sum of the ``rolling downhill'' force ($-\nabla \phi$)
and something proportional to $-\nabla c(x_*)$ to be zero:
\[
  -\nabla \phi(x_*) - \mu \nabla c(x_*) = 0.
\]
The {\em Lagrange multiplier} $\mu$ in this picture represents the
magnitude of the restoring force from the wall balancing the tendency
to roll downhill.

More abstractly, and more generally, suppose that we have a mix of
equality and inequality constraints.  We define
the {\em augmented Lagrangian}
\[
  L(x, \lambda, \mu) = \phi(x) +
    \sum_{i \in \mathcal{E}} \lambda_i c_i(x) +
    \sum_{i \in \mathcal{I}} \mu_i c_i(x).
\]
The {\em Karush-Kuhn-Tucker (KKT) conditions} for $x_*$ to be a
constrained minimizer are
\begin{align*}
  \nabla_x L(x_*) &= 0 \\
  c_i(x_*) &= 0, \quad i \in \mathcal{E}
  & \mbox{equality constraints}\\
  c_i(x_*) & \leq 0, \quad i \in \mathcal{I}
  & \mbox{inequality constraints}\\
  \mu_i & \geq 0, \quad i \in \mathcal{I}
  & \mbox{non-negativity of multipliers}\\
  c_i(x_*) \mu_i &= 0, \quad i \in \mathcal{I}
  & \mbox{complementary slackness}
\end{align*}
where the (negative of) the ``total force'' at $x_*$ is
\[
  \nabla_x L(x_*) = \nabla \phi(x_*) +
    \sum_{i\in \mathcal{E}} \lambda_i \nabla c_i(x_*) +
    \sum_{i\in \mathcal{I}} \mu_i \nabla c_i(x_*).
\]
The complementary slackness condition corresponds to the idea that a
multiplier should be nonzero only if the corresponding constraint is
active (a ``restoring force'' is only present if our test ball
is pushed into a wall).

Like the critical point equation in the unconstrained case, the KKT
conditions define a set of (necessary but not sufficient) nonlinear
algebraic equations that must be satisfied at a minimizer.  Because of
the multipliers, we have {\em more} variables than were present in
the original problem.  However, the Jacobian matrix (KKT matrix)
\[
J = \begin{bmatrix}
  \nabla^2_x L(x_*) & \nabla c \\
  (\nabla c)^T & 0
\end{bmatrix}
\]
has a saddle point structure even when $\nabla^2_x \phi$ is positive
definite.  Also, unlike the penalty and barrier approaches described
before, the Lagrange multiplier approach requires that we figure out
which multipliers are active or not --- an approach that seems to lead
to a combinatorial search in the worst case.

\end{document}
