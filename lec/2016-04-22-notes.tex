\documentclass[12pt, leqno]{article} %% use to set typesize 
\usepackage{amsfonts}
\usepackage{amsmath}
\usepackage{amssymb}
\usepackage{fancyhdr}
\usepackage{hyperref}
\usepackage{tikz}
\usepackage{pgfplots}
\usepackage{listings}

\newcommand{\bbR}{\mathbb{R}}
\newcommand{\bbC}{\mathbb{C}}
\newcommand{\calV}{\mathcal{V}}
\newcommand{\calW}{\mathcal{W}}
\newcommand{\ddiag}{\operatorname{diag}}
\newcommand{\fl}{\operatorname{fl}}
\newcommand{\macheps}{\epsilon_{\mathrm{mach}}}
\newcommand{\matlab}{\textsc{Matlab}}

\newcommand{\hdr}[2]{
  \pagestyle{fancy}
  \lhead{Bindel, Spring 2016}
  \rhead{Numerical Analysis (CS 4220)}
  \fancyfoot{}
  \begin{center}
    {\large{\bf Notes for #1}}
  \end{center}
  \lstset{language=matlab,columns=flexible}  
}


\begin{document}
\hdr{2016-04-22}

\section*{Computing with constraints}

Last time, we discussed different ways of thinking about
solving constrained optimization problems of the form
\[
  \mbox{minimize } \phi(x) \mbox{ s.t.~} x \in \Omega
\]
where the feasible set $\Omega$ is defined by equality and inequality conditions
\[
\Omega = \{ x \in \bbR^n :
  c_i(x) = 0, i \in \mathcal{E} \mbox{ and }
  c_i(x) \leq 0, i \in \mathcal{I} \}.
\]
We will suppose $\phi$ and all the functions $c$ are differentiable.

Last time, we discussed three approaches to re-casting constrained
optimization problems as problems we have already addressed
(unconstrained optimization and nonlinear equation solving).
The approaches were
\begin{itemize}
\item
  {\em Constraint elimination}:
  Find a parameterization $g : \bbR^m \rightarrow \Omega$.
  This is most straightforward for linear equality constraints.
\item
  {\em Barriers and penalties}:
  Add a term to the objective function, dependent on some parameter
  $\mu$.  As $\mu \rightarrow 0$, the unconstrained minima of the
  modified problems converge to the constrained minima of the original/
\item
  {\em Lagrange multipliers}:
  Add new variables (multipliers) corresponding to ``forces'' needed
  to enforce the constraints\footnote{%
    There are other, more mathematical ways of thinking about Lagrange
    multipliers.  I like thinking of forces because it reduces the
    chances that I will make a sign error.
  }.
  The {\em KKT conditions} are a set of nonlinear equations in the
  original unknowns and the multipliers that characterize constrained
  stationary points.
\end{itemize}
While we talked about formulations, we spoke hardly at all about
algorithms.  While there is too little time to do the area any real
justice, we will take today to develop one or two ideas in
particular contexts.

\section*{Quadratic programs with equality constraints}

% Basic setup
% Variables sum to one
% Problems with general sparsity

We begin with a simple case of a quadratic objective and linear
equality constraints:
\begin{align*}
  \phi(x) &= \frac{1}{2} x^T H x - x^T d \\
  c(x) &= A^T x-b = 0,
\end{align*}
where $H \in \bbR^{n \times n}$ is symmetric and positive definite,
$A \in \bbR^{n \times m}$ is full rank with $m < n$, and $b \in \bbR^m$.
Not only are such problems useful in their own right, solvers for
these problems are also helpful building blocks for more sophisticated
problems --- just as minimizing an unconstrained quadratic can be seen
as the starting point for Newton's method for unconstrained optimization.

\subsection*{Constraint elimination (linear constraints)}

As discussed last time, we can write the space of solutions to the
constraint equations in terms of a (non-economy) QR decomposition
of $A$:
\[
  A =
  \begin{bmatrix} Q_1 & Q_2 \end{bmatrix}
  \begin{bmatrix} R_1 \\ 0 \end{bmatrix}
\]
where $Q_2$ is a basis for the null space of $A$.  The set of
solutions satisfying the constraints $Ax = b$ is
\[
  \Omega = \{ u + Q_2 y : y \in \bbR^{(n-m)}, u = Q_1 R_1^{-T} b \};
\]
here $u$ is a {\em particular solution} to the problem.  If we
substitute this parameterization of $\Omega$ into the objective,
we have the unconstrained problem
\[
  \mbox{minimize } \phi(u + Q_2 y).
\]
While we can substitute directly to get a quadratic objective in
terms of $y$, it is easier (and a good exercise in remembering
the chain rule) to compute the stationary equations
\begin{align*}
  0
  &= \nabla_y \phi(u + Q_2 y) 
  = \left(\frac{\partial x}{\partial y}\right)^T \nabla_x \phi(u+Q_2 y) \\
  &= Q_2^T (H (Q_2 y + u) - d) 
  = (Q_2^T H Q_2) y - Q_2^T (d-Hu).
\end{align*}
In general, even if $A$ is sparse, $Q_2$ may be dense, and so even if
$H$ is dense, we find that $Q_2^T H Q_2$ is dense.

\subsection*{Barriers, penalties, and conditioning}

Now consider a penalty formulation of the same equality-constrained
optimization function, where the penalty is quadratic:
\[
  \mbox{minimize } \phi(x) + \frac{1}{2\mu} \|A^T x-b\|^2.
\]
In fact, the augmented objective function is again quadratic, and
the critical point equations are
\[
  (H + \mu^{-1} AA^T) x = d + \mu^{-1} A b.
\]
We can analyze this more readily by changing to the $Q$ basis from
the QR decomposition of $A$ that we saw in the constraint elimination
approach:
\[
\begin{bmatrix}
  Q_1^T H Q_1 + \mu^{-1} R_1 R_1^T & Q_1^T H Q_2 \\
  Q_2^T H Q_1 & Q_2^T H Q_2
\end{bmatrix}
(Q^T x) =
\begin{bmatrix}
  Q_1^T d + \mu^{-1} R_1 b \\
  Q_2^T d
\end{bmatrix}
\]
Taking a Schur complement, we have
\[
(\mu^{-1} R_1 R_1^T + F)(Q_1^T x) = \mu^{-1} R_1 b - g
\]
where
\begin{align*}
  F &= Q_1^T H Q_1 - Q_1^T H Q_2 (Q_2^T H Q_2)^{-1} Q_2^T H Q_1 \\
  g &= [I - Q_1^T H Q_2 (Q_2^T H Q_2)^{-1} Q_2^T] d
\end{align*}
As $\mu \rightarrow 0$, the first row of equations is dominated by the
$\mu^{-1}$ terms, and we are left with
\[
  R_1 R_1^T (Q_1^T x) - R_1 b \rightarrow 0
\]
i.e.~$Q_1 Q_1^T x$ is converging to $u = Q_1 R_1^{-T} b$, the
particular solution that we saw in the case of constraint elimination.
Plugging this behavior into the second equation gives
\[
  (Q_2^T H Q_2) (Q_2^T x) - Q_2^T (d-Hu) \rightarrow 0,
\]
i.e.~$Q_2^T x$ asymptotically behaves like $y$ in the previous
example.  We need large $\mu$ to get good results if the constraints
are ill-posed or if $Q_2^T H Q_2$ is close to singular.  But in
general the condition number scales like $O(\mu^{-1})$, and so large
values of $\mu$ correspond to problems that are numerically
unattractive.

\subsection*{Lagrange multipliers and KKT systems}

% Basic setup -- same as before
% Enforcing constraints (quadratic penalty)
% Limiting behavior

The KKT conditions for our equality-constrained problem say that the
gradient of
\[
  L(x,\lambda) = \phi(x) + \lambda^T (A^T x-b)
\]
should be zero.  In matrix form, the KKT system (saddle point system)
\[
  \begin{bmatrix}
    H & A \\
    A^T & 0
  \end{bmatrix}
  \begin{bmatrix} x \\ \lambda \end{bmatrix} =
  \begin{bmatrix} d \\ b \end{bmatrix}.
\]
If $A$ and $H$ are well-conditioned, then so is this system,
so there is no bad numerical behavior.  The system also retains
whatever sparsity was present in the original system matrices
$H$ and $A$.  However, adding the Lagrange multipliers not only
increases the number of variables, but the extended system lacks
any positive definiteness that $H$ may have.

The KKT system is closely related to the penalty formulation that
we saw in the previous subsection, in that if we use Gaussian
elimination to remove the variable $\lambda$ in
\[
  \begin{bmatrix}
    H & A \\
    A^T & -\mu I
  \end{bmatrix}
  \begin{bmatrix} \hat{x} \\ \lambda \end{bmatrix} =
  \begin{bmatrix} d \\ b \end{bmatrix},
\]
we have the Schur complement system
\[
  (H+\mu^{-1} AA^T) \hat{x} = d + \mu^{-1} A b,
\]
which is identical to the stationary point condition for the
quadratically penalized objective.

\subsection*{Other approaches}

Of course, the three approaches we have sketched above are not the
only methods!  For example, augmented Lagrangian methods combine a
penalty approach with an approximate Lagrange multiplier, alternately
solving an unconstrained problem and updating a set of multiplier
estimates.  Other methods (e.g.~projected gradient descent) are
available if there is a cheap way to project a proposed solution to
the closest point on the constraint set.  And one can apply the same
types of inexact iterations and quasi-Newton ideas we saw in the
context of unconstrained optimization.

\section*{Inequality constraints}

% Constraint shuffling and QR updating
% Barriers and penalties
% Active set methods

Problems with inequality constraints can be reduced to problems with
{\em equality} constraints if we can only figure out which constraints
are active at the solution.  We use two main strategies to tackle
this task:
\begin{itemize}
\item {\em Active set} methods guess which constraints are active, then
  solve an equality-constrained problem.  If the solution satisfies
  the KKT conditions, we are done.  Otherwise, we update the guess of
  the active set by looking for constraint violations or negative
  multipliers.  The {\em simplex method} for linear programs is a
  famous active set method.  The difficulty with these methods is that
  it may take many iterations before we arrive on the correct active set.
\item {\em Interior point} methods take advantage of the fact that
  barrier formulations do not require prior knowledge of the active
  constraints; rather, the solutions converge to an appropriate
  boundary point as one changes the boundary.
\end{itemize}
Between the two, active set methods often have an edge when it is easy
to find a good guess for the constraints.  Active set methods are
great for families of related problems, because they can be ``warm
started'' with an initial guess for what constraints will be active
and for the solution.  Many strong modern solvers are based on
sequential quadratic programming, a Newton-like method in which the
model problems are linearly-constrained quadratic programs that are
solved by an active set iteration.  In contrast to active set methods,
interior point methods spend fewer iterations sorting out which
constraints are active, but each iteration may require more work.

\section*{Solving least squares on a simplex}

% Lagrange multipliers for normalization
% Simplex for other

A couple years ago, I was faced with a machine learning task in which
a key subproblem involved a large number of small simplex-constrained least
squares problems:
\[
  \mbox{minimize } \frac{1}{2} \|Ax-b\|^2 \mbox{ s.t. } x \geq 0
  \mbox{ and } \sum_i x_i = 1.
\]
The matrix $A$ remained constant across all instances of the problem;
only the right hand side $b$ changed.  I did not have an appropriate
specialized solver at hand, and the general-purpose solvers at my
disposal were all a bit heavy-weight for the context where I wanted
them.  I followed a combined strategy that used several of the ideas
we have discussed this week, and wanted to close the lecture with a
few of them.

\subsection*{Problem reduction and initial guess}

First, recall that if $A = QR$ is an economy QR decomposition, then
\[
  \|Ax-b\|^2 = \|Rx-Q^T b\|^2 + \|(I-QQ^T)b\|^2.
\]
This decomposition of the squared residual norm has nothing to do with
how we constraint $x$.  Hence, we can rewrite the simplex-constrained
problem as
\[
  \mbox{minimize } \frac{1}{2} \|Rx-\tilde{b}\|^2 \mbox{ s.t. } x \geq 0
  \mbox{ and } \sum_i x_i = 1,
\]
where $\tilde{b} \equiv Q^T b$.

As an initial guess, we project the solution to the unconstrained
problem onto the simplex (i.e.~the set $\Omega$ of points with
non-negative coordinates that sum to one).  I found how to do this via
Google; it's in Figure 1 in an ICML 2008 paper by Duchi {\em et al}.
With this initial guess, we have a proposed active set for
the first step of the algorithm.

\subsection*{Normalization constraint}

Suppose $\mathcal{J}$ is the set of variables for which the
non-negativity constraint is inactive.  That is, we guess
that a solution will satisfy
\[
\begin{cases}
  x_j > 0, & j \in \mathcal{J} \\
  x_j = 0, & j \not \in \mathcal{J}
\end{cases}
\]
We will assume (without loss of generality, as we shall see) that
$\mathcal{J} = \{ 1, 2, \ldots, \|\mathcal{J}|\}$; in this case,
we want to solve the equality-constrained subproblem for a step
from the current solution estimate $x_{\mathcal{J}}$ to the next one:
\[
  \mbox{minimize } \frac{1}{2} \|\bar{R} p - f\|^2
  \mbox{ s.t.~} \sum_{j \in \mathcal{J}} p_j = 0,
\]
where $\bar{R} = R(\mathcal{J}, \mathcal{J})$ is a leading submatrix
of $R$ and $f = b(\mathcal{J})-\bar{R} x_{\mathcal{J}}$ is the leading
subvector of the current residual.
Were it not for the equality constraint, we could solve
this by back-substitution.  To enforce the equality constraint, we
add a Lagrange multiplier, leading to the KKT system
\[
\begin{bmatrix}
  \bar{R}^T \bar{R} & e \\
  e^T & 0
\end{bmatrix}
\begin{bmatrix}
  p \\ \lambda
\end{bmatrix} =
\begin{bmatrix}
  \bar{R}^T f \\ 0
\end{bmatrix}.
\]
We can multiply the first block row by $\bar{R}^{-T}$ and the first block
column by $\bar{R}^{-1}$ to find
\[
\begin{bmatrix}
  I & w \\
  w^T & 0
\end{bmatrix}
\begin{bmatrix}
  \bar{R} p \\ \lambda
\end{bmatrix} =
\begin{bmatrix}
  f \\ 0
\end{bmatrix}
\]
where $w \equiv \bar{R}^{-T} e$.  Gaussian elimination on this system
gives
\begin{align*}
  \lambda &= \frac{w^T f}{w^T w} \\
  \bar{R} x_{\mathcal{J}} &= f-\lambda w
\end{align*}
Hence, solving the constrained system involves one additional
triangular solve (to compute $w$) and a few dot products beyond
the usual triangular solve we would use for the unconstrained problem.

Once we have a direction, we take a step in that direction.  If taking
a full step violates a constraint, we take the longest partial step
that does not violate any constraints, and we add the constraint that
controls the step size to the proposed active set.  Otherwise, after
taking a full step, we check whether all the constraints marked as
active really should be active.  If one or more constraints should not
be active, we remove the one that is ``worst'' (in the sense of
the most negative Lagrange multiplier) from the active set.

\subsection*{Back to numerical linear algebra}

We like the convention that the first columns of $\bar{R}$ are
correspond to the active elements.  To maintain this convention,
we need primitives to update the $R$ factor in a QR factorization
after transposing two columns of the original matrix.  That is,
we want to be able to
\begin{itemize}
\item Start with a triangular matrix $R$
\item Swap columns $j$ and $j' > j$ to get a new matrix $\tilde{R}$
\item Apply orthogonal transformations from the right to get
  a new upper triangular matrix $R'$.
\end{itemize}
Because this happens at every step of the nonnegative least squares
iteration, we want it to run fast.  We do this in the follosing steps:
\begin{itemize}
\item
  $\tilde{R}$ is not triangular because there are too many nonzeros in
  column $j$.  To eliminate these nonzeros, we make an ``upward pass''
  where we apply a sequence of Givens rotations on row pairs
  $(j'-1,j'), (j'-2,j'-1), \ldots, (j+1,j+2)$ to introduce zeros in
  column $j$ below the first subdiagonal.
\item
  After applying the first set of Givens rotations, our matrix is now
  in upper Hessenberg form.  To restore to triangular form, we make
  a ``downward pass'' where we use Givens rotations on row pairs $(j,j+1)$
  through $(j'-1,j')$ to zero out the subdiagonal entries in columns $j$
  through $j'$.
\end{itemize}
The cost of applying a Givens rotation across the rows of the matrix
is $O(n)$, so the total cost of this two-sweep update is $O(n(j'-j))$.

The Givens-based matrix update/downdate technology lets us compute
each step in the active set iteration in at most $O(n^2)$ time --- and
potentially less.  Because we are using the same $R$ matrix for many
distinct problems, we do not even require an $O(n^3)$ factorization
for each new right hand side!  In contrast, a solver that used
barriers or penalties would usually require $O(n^3)$ time per
iteration.  Consequently, a little care in our choice of algorithms
--- and in our linear algebra implementations --- can lead to a big
performance boost.

\end{document}
