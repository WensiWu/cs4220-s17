\documentclass[12pt, leqno]{article} %% use to set typesize
\usepackage[utf8]{inputenc}
\usepackage[russian]{babel}

\usepackage{amsfonts}
\usepackage{amsmath}
\usepackage{amssymb}
\usepackage{fancyhdr}
\usepackage{hyperref}
\usepackage{tikz}
\usepackage{pgfplots}
\usepackage{listings}

\newcommand{\bbR}{\mathbb{R}}
\newcommand{\bbC}{\mathbb{C}}
\newcommand{\calV}{\mathcal{V}}
\newcommand{\calW}{\mathcal{W}}
\newcommand{\ddiag}{\operatorname{diag}}
\newcommand{\fl}{\operatorname{fl}}
\newcommand{\macheps}{\epsilon_{\mathrm{mach}}}
\newcommand{\matlab}{\textsc{Matlab}}

\newcommand{\hdr}[2]{
  \pagestyle{fancy}
  \lhead{Bindel, Spring 2016}
  \rhead{Numerical Analysis (CS 4220)}
  \fancyfoot{}
  \begin{center}
    {\large{\bf Notes for #1}}
  \end{center}
  \lstset{language=matlab,columns=flexible}  
}


\begin{document}
\hdr{2016-04-29}

\section*{Parameter Fitting}

Our focus this week is on optimization problems and nonlinear
equations with specialized structure.  Last lecture, we touched on
some of the many places where spectral methods make sense.  This time,
we consider nonlinear least squares problems.  Such problems are
common in cases where we want to fit model parameters to (possibly
inconsistent) data, and there are a variety of methods to treat
them, taking advantage of different types of structure in the problem.

\section*{Gauss-Newton}

Suppose $F : \bbR^n \rightarrow \bbR^m$ with $m > n$, and let
\[
  \phi(x) = \frac{1}{2} \|F(x)\|^2.
\]
We consider the nonlinear least squares problem of minimizing $\phi$.
The most obvious approach, supposing we are willing to compute lots
of derivatives, is to apply Newton iteration.  To do this, we need
not only function values, but also gradients and Hessians:
\begin{align*}
  \phi(x) &= \frac{1}{2} \sum_{i=1}^m F_i(x)^2 \\
  \phi'(x) &= \sum_{i=1}^m F_i(x) F_i'(x) \\
  \phi''(x) &= \sum_{i=1}^m F_i'(x)^T F_i'(x) + F_i(x) F_i''(x)
\end{align*}
In terms of the residual $F(x) \in \bbR^m$ and the Jacobian
$J(x) = F'(x) \in \bbR^{m \times n}$, 
\begin{align*}
  \nabla \phi(x) &= J^T F \\
  \phi''(x) = H &= J^T J + \sum_{i=1}^m F_i(x) F_i''(x).
\end{align*}
That second Hessian term is awkward, as it involves computing a
Hessian of each component of $F$ in terms -- that's a lot of
derivatives!  Fortunately, if the solution has small residual
(each $F_i$ is small at the optimum), then the second term in
the Hessian also should not matter very much.  If we 
drop the second term, we get the {\em Gauss-Newton iteration}
\[
  x^{k+1} = x^k - (J^T J)^{-1} J^T F(x^k)
\]
which, having absorbed the first half of the semester,
we identify as
\[
  x^{k+1} = x^k - J^{\dagger} F(x^k)
\]
where $J^\dagger$ is the Moore-Penrose pseudo-inverse (a.k.a.~the
solution operator for a linear least squares problem).  Equivalently,
we say that $x^{k+1} = x^k + p^k$ where $p^k$ is the solution to
\[
  \mbox{minimize } \|F(x^k) + J(x^k) p^k\|^2.
\]
That is, the Gauss-Newton iteration can be seen either as dropping an
inconvenient term in a Newton iteration, or as successive miminizing
linear least-squares models of the nonlinear least-squares problem.

\section*{Convergence of Gauss-Newton and Beyond}

Unlike Newton, Gauss-Newton has the attractive feature that (assuming
$J$ is full rank) it always produces a descent direction at any point
that is not already a stationary point:
\[
  -\nabla \phi(x)^T p = -(J^T F)^T (J^\dagger F) = -F^T \Pi F.
\]
where $\Pi = J J^\dagger$ is the orthogonal projector onto the range
space of $J$.

What of the asymptotic convergence behavior?  With some work,
we can get an error iteration
\[
  \|e^{\mathrm{new}}\| \lesssim \frac{M
    \|F(x_*)\|}{\sigma_{\min}(J(x_*))^2} \|e^{\mathrm{old}}\|,
\]
where $M$ is a local Lipschitz constant for $J$.  This shows local
linear convergence if $M \|F(x_*)\| < \sigma_{\min}(J(x_*))^2$.  But
if $F(x_*)$ is large relative to the small singular values of
$J(x_*)$, the iteration {\em will not converge without safeguards}.

We can restore guaranteed convergence by a line search.  Alternately,
we can use a trust region approach or regularization term in the
linear least squares problem for the update; this leads to the
Levenberg-Marquardt iteration.

\section*{Iteratively Reweighted Least Squares}

Many nonlinear least squares problems involve a nonlinear loss
function applied elementwise to a linear residual.  We assume
the loss is positive, and write it as $\ell(r_i) = f(r_i)^2$,
so the problem becomes
\[
\mbox{minimize } \frac{1}{2} \sum_i \ell(r_i)
= \frac{1}{2} \sum_i f(r_i)^2, \quad r = Ax-b.
\]
This is a nonlinear least squares problem; if we apply the
Gauss-Newton idea, we have steps of the form
\[
  \mbox{minimize } \|f(r) + \operatorname{diag}(f'(r)) A p\|^2.
\]
That is, each step is the solution to a weighted least squares problem
\[
  \mbox{minimize } \|W(Ap-b)\|^2
\]
where the diagonal weight matrix $W$ has $w_{ii} = f'(r_i)$ and
$b_i = -f(r_i)/f'(r_i)$.  Because the weights vary at each step,
this is known as an {\em iteratively reweighted least squares} (IRLS)
method.

The iteratively reweighted least squares idea also appears in
statistics under the guise of {\em Fisher scoring} for computing
maximum likelihood estimates.  As with Gauss-Newton, the idea behind
Fisher scoring is to replace a hard-to-manage Hessian with something
simpler (the distributional expected value of the Hessian as opposed
to the Hessian derived from the sample).  Unfortunately, the
literature is often slightly confusing in that many authors fail to
distinguish an exact Newton iteration on the scoring function
from the approximate Newton iteration in Fisher's approach.

\section*{Variable Projection}

% - Parameter fitting and nonlinear least squares
% - From Newton to Gauss-Newton
% - Gauss-Newton via models
% - Iteratively reweighted least squares
% - Variable projection

The case of a linear model with a non-quadratic loss function leads to
one special class of nonlinear least squares.  Another common special
case is when a model depends linearly on some parameters and
nonlinearly on others.  For these problems, {\em variable projection}
is the general strategy of eliminating the variables on which a least
squares problem depends linearly.

As an example, consider the problem
\[
  \mbox{minimize } \phi(x,y) = \frac{1}{2} \|A(y) x-b\|^2
\]
where $A : \bbR^p \rightarrow \bbR^{m \times n}$ depends (possibly
nonlinearly) on $y$, but the $x$ variables only enter the problem
linearly.  The {\em variable projection} approach involves
eliminating the $x$ variables from the equation:
\[
  \mbox{minimize } \phi(x(y), y) =
  \frac{1}{2} \|r(y)\|^2,
  \quad r(y) = A(y) x(y) - b,
  \quad x(y) = A(y)^\dagger b.
\]
The variation of $\|r\|^2/2$ is $r^T \delta r$ where
\[
  \delta r = A (\delta x) + (\delta A) x;
\]
and because $A^T r = 0$ (normal equations), we have
\[
  r^T \delta r = r^T (\delta A) x.
\]
One may undertake second derivatives as an exercise in algebraic
fortitude; alternately, we may apply BFGS or similar methods.
Either way, we are now left with a smaller optimization problem
involving the $y$ variables alone.

\end{document}
